\documentclass[10pt]{article}
\input{style/coursHeadings}
\input{style/programHeadings}
\input{style/macros_SII}
\input{style/macros_Titres}
\input{style/macros_Frames}

%Si le boolen xp est vrai : compilation pour xabi
%Sinon compilation Damien
\newboolean{xp}
\setboolean{xp}{true}

\newboolean{prof}
\setboolean{prof}{false}

\newboolean{td}
\setboolean{td}{true}


\usepackage[%
    pdftitle={CI 06 : Stat - Ch 02 : PFS},
    pdfauthor={Xavier Pessoles},
    colorlinks=true,
    linkcolor=blue,
    citecolor=magenta]{hyperref}


\def\discipline{Sciences Industrielles de l'Ingénieur}
\def\xxtitre{\ifthenelse{\boolean{xp}}{
CI 6 : Étude du comportement statique des systèmes}{
Chapitre  -- }}

\def\xxsoustitre{\ifthenelse{\boolean{xp}}{
Chapitre 2 -- Principe Fondamental de la Statique}{
Partie  -- }}

\def\xxauteur{\ifthenelse{\boolean{xp}}{
Xavier \textsc{Pessoles} \\ 2013 -- 2014}{
}}

\def\xxpied{\ifthenelse{\boolean{xp}}{
CI 6 : Statique\\
Chapitre 2 : PFS -- TD 3}{
\xxtitre}}

\def\xxcathegorie{\ifthenelse{\boolean{xp}}{
2013 -- 2014 \\
Xavier \textsc{Pessoles}}{}}





%---------------------------------------------------------------------------


\begin{document}

\ifthenelse{\boolean{xp}}{\input{style/enteteXP}}{\input{style/enteteDI}}

\begin{center}
\large{\textsc{Travaux Dirigés}}
\end{center}

\begin{flushright}
\textit{Ressources de Florestan Mathurin.}
\end{flushright}



\section*{Système de positionnement de radar}

\ifthenelse{\boolean{prof}}{}{

\begin{minipage}[c]{.45\linewidth}
Afin d'assurer la sécurité des personnes chargées de déminer les terrains militaires, les ingénieurs ont imaginé d'intégrer un détecteur de métaux à un véhicule blindé. Grâce à ce système, les démineurs peuvent rester à l'abri dans le véhicule ou cas où la mine venait à exploser.
\end{minipage}\hfill
\begin{minipage}[c]{.5\linewidth}
\begin{center}
\includegraphics[width=\textwidth]{images/img1}
\end{center}
\end{minipage}

\vspace{.5cm}

On donne ci-dessous le schéma d'architecture de la solution retenue ainsi qu'un extrait du cahier des charges fonctionnel.

\begin{center}
\includegraphics[width=.9\textwidth]{images/img2}
\end{center}

Le balayage par l'antenne de la zone à nettoyer est réalisé grâce à la rotation d'axe $(AB)$ de la pièce 2 actionnée par un vérin hydraulique 3--4. L'antenne du détecteur de métaux est fixée sur un support permettant une double rotation suivant deux axes perpendiculaires de façon à ce qu'elle puisse toujours rester parallèle au sol. Ces deux rotations sont générées par deux moteurs électriques. Le support $S$ est fixé sur le châssis du véhicule par un vérin.

L'objectif de cette étude est de vérifier que le positionneur de radar permet de satisfaire ou non les critères de balayage angulaire à effectuer et de masse de radar à déplacer FS1. 

Hypothèses de modélisation : 
\begin{itemize}
\item la pesanteur n'agit que sur l'antenne de son centre $G$; 
\item la base $(\vect{x_0}, \vect{y_0}, \vect{z_0})$ est liée au sol. Le sol est en pente selon un angle $\alpha=15^o$ par rapport à la verticale ascendante de la base $(\vect{x_g}= \vect{x_0}, \vect{y_g},\vect{z_g})$;
\item la base $(\vect{x_1}, \vect{y_1}, \vect{z_1}=\vect{z_0})$ est liée au véhicule 1 qui se déplace sur le sol selon un angle $\beta$ ($0\leq \beta \leq 2\pi$);
\item la base $(\vect{x_2}, \vect{y_2}, \vect{z_2}=\vect{z_1})$  est liée au bras 2 du positionneur de radar. La position du bras 2 par rapport à la voiture est paramétrée par l'angle $\theta$;
\item la section du vérin 3--4 est notée $S$; 
\item la pression dans le vérin est notée $p$ et est supposée constante;
\item toutes les liaisons sont supposées parfaites.
\end{itemize}


\begin{center}
\includegraphics[width=.9\textwidth]{images/img3}
\end{center}
}
%\subsection*{Etude du critère de masse à déplacer}

\subsection*{Étude du critère de masse à déplacer}
\ifthenelse{\boolean{prof}}{}{
On retient le modèle 3D suivant. Le solide 2 comprend l'antenne et son support orientable. 


\begin{center}
\includegraphics[width=.9\textwidth]{images/img4}
\end{center}
}

\subparagraph{}
\textit{Établir le graphe de structure du système. Indiquer sur ce graphe le nombre d'inconnues des torseurs d'actions mécaniques transmissibles par chacune des liaisons.}
\ifthenelse{\boolean{prof}}{
\begin{corrige}
\begin{center}
\includegraphics[width=.6\textwidth]{images/img1_cor}
\end{center}
\end{corrige}}{}


\ifthenelse{\boolean{prof}}{}{
L'action mécanique transmissible par la liaison rotule $A$ est modélisée par le torseur suivant :
$$
\left\{
F_{\text{rotule 1}\rightarrow 2} 
\right\}=
\left\{
\begin{array}{cc}
X_{12} & 0 \\
Y_{12} & 0 \\
Z_{12} & 0 \\
\end{array}
\right\}_{A,B_1}
$$

L'action mécanique transmissible par la liaison linéaire annulaire en $B$ d'axe $(B,\vect{z_1})$ est modélisée par le torseur suivant :
$$
\left\{
F_{\text{lin annulaire }1\rightarrow 2} 
\right\}=
\left\{
\begin{array}{cc}
X'_{12} & 0 \\
Y'_{12} & 0 \\
0 & 0 \\
\end{array}
\right\}_{B,B_1}
$$

L'action mécanique de la pesanteur sur le solide 2 est modélisée par le torseur suivant :
$$
\left\{
F_{\text{pesanteur }\rightarrow 2} 
\right\}=
\left\{
\begin{array}{cc}
- P \vect{z_g} \\
\vect{0} \\
\end{array}
\right\}_{G}
$$
}
\subparagraph{}
\textit{Donner la forme des torseurs d'actions transmissibles des liaisons 1--3, 3--4 et 2--4 dans la base 3 conformément à la notation imposée ci-dessous.}
\ifthenelse{\boolean{prof}}{
\begin{corrige}
La liaison entre 1 et 3 est une liaison rotule de centre $C$: 

$$
\left\{
F_{1 \rightarrow 3} 
\right\}=
\left\{
\begin{array}{cc}
X_{13} & 0 \\
Y_{13} & 0 \\
Z_{13} & 0 \\
\end{array}
\right\}_{C,B_3}
$$

La liaison entre 3 et 4 est une liaison pivot glissant de de centre $F$ et d'axe $\vect{x_3}$: 

$$
\left\{
F_{4 \rightarrow 3} 
\right\}=
\left\{
\begin{array}{cc}
0 & 0 \\
Y_{43} & M_{43} \\
Z_{43} & N_{43} \\
\end{array}
\right\}_{F,B_3}
$$

La liaison entre 2 et 4 est une liaison rotule de de centre $D$ : 

$$
\left\{
F_{2 \rightarrow 4} 
\right\}=
\left\{
\begin{array}{cc}
X_{24} & 0 \\
Y_{24} & 0 \\
Z_{24} & 0 \\
\end{array}
\right\}_{D,B_3}
$$
\end{corrige}}{}

Action mécanique exercée par le solide $i$ sir le solide $j$ au point $P$ dans la base 3 :
$$
\left\{
F_{j \rightarrow i} 
\right\}=
\left\{
\begin{array}{cc}
X_{ij} & L_{ij} \\
Y_{ij} & M_{ij} \\
Z_{ij} & N_{ij} \\
\end{array}
\right\}_{P,B_3}
\quad \text{avec} \quad
\vect{R_{i\rightarrow j}} = X_{ij}\vect{x_3}+Y_{ij}\vect{y_3}+Z_{ij}\vect{z_3} 
\quad \text{et} \quad
\vect{M_{P, i\rightarrow j}} = L_{ij}\vect{x_3}+M_{ij}\vect{y_3}+N_{ij}\vect{z_3} 
$$

\subparagraph{}
\textit{Déterminer les directions de $\vect{R_{1\rightarrow 3}}$ et de $\vect{R_{2\rightarrow 4}}$ puis en déduire des simplifications dans les torseurs précédents.}
\ifthenelse{\boolean{prof}}{
\begin{corrige}
On isole les solides $3$ et $4$. Cet ensemble est soumis à 2 forces. D'après le PFS, ces deux forces ont même norme, même direction et sens opposé. En conséquence, $X_{13}+X_{24}=0$ et $Y_{13}=Z_{13}=Y_{24}=Z_{24}=0$.

$$
\left\{
F_{1 \rightarrow 3} 
\right\}=
\left\{
\begin{array}{cc}
X_{13} & 0 \\
0 & 0 \\
0 & 0 \\
\end{array}
\right\}_{C,B_3}
\quad 
\left\{
F_{2 \rightarrow 4} 
\right\}=
\left\{
\begin{array}{cc}
X_{24} & 0 \\
0 & 0 \\
0 & 0 \\
\end{array}
\right\}_{D,B_3}
$$
\end{corrige}}{}

\subparagraph{}
\textit{Déterminer l'expression de $||\vect{R_{2\rightarrow 4}}||$ en fonction de $p$ et $S$.}
\ifthenelse{\boolean{prof}}{
\begin{corrige}
On a $||\vect{R_{2\rightarrow 4}}||=pS$ 
\end{corrige}}{}

\subparagraph{}
\textit{En isolant le solide $2$ et en utilisant le théorème du moment statique au point $A$ projetée sur $\vect{z_2}$, déterminer l'expression de $X_{42}$ en fonction de $P$ et des paramètres géométriques utiles.}
\ifthenelse{\boolean{prof}}{
\begin{corrige}
On isole le solide 2. 

On fait le bilan des actions mécaniques extérieures au point $A$ :
Liaison rotule en $A$ : 
$$
\left\{
F_{1 \rightarrow 2} 
\right\}=
\left\{
\begin{array}{cc}
X_{12} & 0 \\
Y_{12} & 0 \\
Z_{12} & 0 \\
\end{array}
\right\}_{A,B_2}
$$



Liaison linéaire annulaire au point $B$ :
$$
\left\{
F_{1 \rightarrow 2} 
\right\}=
\left\{
\begin{array}{cc}
0 & 0 \\
Y'_{12} & 0 \\
Z'_{12} & 0 \\
\end{array}
\right\}_{B,B_2}
=
\left\{
\begin{array}{cc}
0 &  -\\
Y'_{12} &  -\\
Z'_{12} &  0\\
\end{array}
\right\}_{A,B_2}
$$


Liaison rotule au point $D$ :
$$
\left\{
F_{4 \rightarrow 2} 
\right\}
=
\left\{
\begin{array}{cc}
X_{42}  & 0 \\
0 & 0 \\
0 & 0 \\
\end{array}
\right\}_{D,B_2}
=
\left\{
\begin{array}{cc}
X_{42}  & -\\
0 & -\\
0 & 0 \\
\end{array}
\right\}_{A,B_2}
$$


Pesanteur en $G$ :
$$
\left\{
F_{pes \rightarrow 2} 
\right\}
=
\left\{
\begin{array}{cc}
-P  & 0 \\
0 & 0 \\
0 & 0 \\
\end{array}
\right\}_{G,B_2}
=
\left\{
\begin{array}{cc}
-P  & -\\
0 & -  \\
0 &  \\
\end{array}
\right\}_{A,B_2}
$$
\end{corrige}}{}

On donne : $\vect{BA}=180\vect{z_0}$, $\vect{BA}=-230\vect{y_1}$, $\vect{AD}=710\vect{x_2}$, $\vect{AG}=1200\vect{x_2}-270\vect{z_2}$ où les dimensions sont en mm. $P=400N$, $S=28\cdot 10^{-3}m^2$.

\subparagraph{}
\textit{Calculer la valeur de la pression $p$ maximale pour la position $\theta=0^o$ (et donc $\gamma = 18^o$) et $\beta = 0^o$ avec les valeurs numériques données.}
\ifthenelse{\boolean{prof}}{
\begin{corrige}
\end{corrige}}{}


\subparagraph{}
\textit{Le vérin utilisé peut supporter une pression de 10 bars maximum. Conclure quant à la capacité du système à satisfaire le critère de masse de radar à déplacer.}
\ifthenelse{\boolean{prof}}{
\begin{corrige}
\end{corrige}}{}







\end{document}