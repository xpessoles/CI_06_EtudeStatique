\documentclass[10pt]{article}
\input{style/coursHeadings}
\input{style/programHeadings}
\input{style/macros_SII}
\input{style/macros_Titres}
\input{style/macros_Frames}

%Si le boolen xp est vrai : compilation pour xabi
%Sinon compilation Damien
\newboolean{xp}
\setboolean{xp}{true}

\newboolean{prof}
\setboolean{prof}{false}

\newboolean{td}
\setboolean{td}{true}


\usepackage[%
    pdftitle={CI 06 : Stat - Ch 02 : PFS},
    pdfauthor={Xavier Pessoles},
    colorlinks=true,
    linkcolor=blue,
    citecolor=magenta]{hyperref}


\def\discipline{Sciences Industrielles de l'Ingénieur}
\def\xxtitre{\ifthenelse{\boolean{xp}}{
CI 6 : Étude du comportement statique des systèmes}{
Chapitre  -- }}

\def\xxsoustitre{\ifthenelse{\boolean{xp}}{
Chapitre 2 -- Principe Fondamental de la Statique}{
Partie  -- }}

\def\xxauteur{\ifthenelse{\boolean{xp}}{
Xavier \textsc{Pessoles} \\ 2013 -- 2014}{
}}

\def\xxpied{\ifthenelse{\boolean{xp}}{
CI 6 : Statique\\
Chapitre 2 : PFS -- TD 5}{
\xxtitre}}

\def\xxcathegorie{\ifthenelse{\boolean{xp}}{
2013 -- 2014 \\
Xavier \textsc{Pessoles}}{}}





%---------------------------------------------------------------------------


\begin{document}

\ifthenelse{\boolean{xp}}{\input{style/enteteXP}}{\input{style/enteteDI}}

\begin{center}
\large{\textsc{Travaux Dirigés}}
\end{center}

\begin{flushright}
\textit{Ressources de Stéphane Genouël.}
\end{flushright}



\section*{Bride hydraulique}

\subsection*{Mise en situation}
Le système étudié a pour fonction de brider (bloquer) des pièces 
sur une table de machine-outil afin de les usiner par la suite. 

 L’alimentation en énergie hydraulique 
permet la sortie de l’ensemble piston-tige 4 
qui fait pivoter le levier 7 par rapport au 
corps 1 et permet ainsi de plaquer la pièce 
à usiner sur la table de la machine-outil à 
l’aide de la vis 8 solidaire du levier 7. 
Un ressort 5, comprimé lors de la phase de 
bridage, permet la rentrée de l’ensemble 
piston-tige 4 lorsque la bride n’est plus 
alimentée en énergie hydraulique et libère 
ainsi la pièce usinée.
 
 \begin{obj}
 Déterminer la valeur minimale 
$p$ de la pression d’alimentation pour respecter l'exigence d'un effort presseur minimal de 4000 N.  

 \end{obj}
 
 \subsection*{Hypothèses}
 
\begin{itemize}
\item Les liaisons sont considérées comme parfaites. 
\item L’action de la pesanteur sur les pièces est négligée par rapport aux autres actions mécaniques. 
\item Le système est en équilibre en phase de bridage dans une position pour laquelle : 
\begin{itemize}
\item le contact entre la vis 8 et la pièce à usiner est ponctuel en J de normale $\vect{y}$; 
\item le contact entre le piston 4 et le levier 7 est ponctuel en I de normale $\vect{y}$; 
\item Il n’y a pas de mouvement relatif entre 10 et 11. 
\end{itemize}
\end{itemize}

 \subsection*{Données}
\begin{itemize}
\item Ressort : 
\begin{itemize}
\item longueur à vide $L_0 = 20\; mm$ ; 
\item longueur dans la position étudiée $L = 16\; mm$; 
\item raideur : $k=10\, N/mm$.
\end{itemize}
\item $\vect{KJ}\cdot\vect{x}=a=-32$;
\item $\vect{KI}\cdot\vect{x}=b= 33$ (distance en mm);
\item piston : rayon $R=30\; mm$.\
\end{itemize}
   \subsection*{Travail demandé}
  \subparagraph{}
 \textit{}
 
   \subparagraph{}
 \textit{Réaliser le graphe de structure, puis compléter-le en vue d’une étude de statique. }
   \subparagraph{}
 \textit{Déterminer, en appliquant le Principe Fondamental de la Statique à \{7, 8, 9\} au point $K$, les 
six équations scalaires liant les composantes d’actions mécaniques et les dimensions du 
système. En déduire l’expression de $Y_{4\rightarrow 7}$ en fonction de l'effort presseur $F$ et des dimensions du système.}
 
   \subparagraph{}
 \textit{Déterminer, en appliquant le Principe Fondamental de la Statique à \{4\} au point I, les six 
équations scalaires liant les composantes d’actions mécaniques et les dimensions du 
système. En déduire l’expression de p en fonction de l’effort presseur F , de la raideur 
k et 
des dimensions du système. 
}
   \subparagraph{}
 \textit{En déduire la valeur minimale de la pression 
p permettant le respect de l'objectif.}


\begin{center}
\includegraphics[width=.9\textwidth]{images/Fi01}
\end{center}
\end{document}



