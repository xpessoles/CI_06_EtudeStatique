\documentclass[10pt]{article}
\input{style/coursHeadings}
\input{style/programHeadings}
\input{style/macros_SII}
\input{style/macros_Titres}
\input{style/macros_Frames}

%Si le boolen xp est vrai : compilation pour xabi
%Sinon compilation Damien
\newboolean{xp}
\setboolean{xp}{true}

\newboolean{prof}
\setboolean{prof}{false}

\usepackage[%
    pdftitle={CI 06 : Stat - Modélisation des AM},
    pdfauthor={Xavier Pessoles},
    colorlinks=true,
    linkcolor=blue,
    citecolor=magenta]{hyperref}


\def\discipline{Sciences Industrielles de l'Ingénieur}
\def\xxtitre{\ifthenelse{\boolean{xp}}{
CI 06 : Étude du comportement statique des systèmes}{}}

\def\xxsoustitre{\ifthenelse{\boolean{xp}}{
Chapitre 1 -- Modélisation des Actions Mécaniques}{
Partie  -- }}

\def\xxauteur{\ifthenelse{\boolean{xp}}{
Xavier \textsc{Pessoles}}{}}

\def\xxpied{\ifthenelse{\boolean{xp}}{
CI 06 : Statique\\
Ch. 1 : Modélisation des AM -- TD -- Géométrie des Masses}{
\xxtitre}}

\def\xxcathegorie{\ifthenelse{\boolean{xp}}{
2013 -- 2014 \\
Xavier \textsc{Pessoles}}{}}





%---------------------------------------------------------------------------


\begin{document}

\ifthenelse{\boolean{xp}}{\input{style/enteteXP}}{\input{style/enteteDI}}

\begin{center}
\Large{\textsc{Exercices d'application : Modèle global -- Calcul de moment dans le plan}}
\end{center}

\begin{flushright}
\textit{D'après ressources de G. Himmelspach.}
\end{flushright}
\vspace{.5cm}

\subsection*{Exercice 1 -- Panneau soumis à l'action du vent}
\setcounter{subparagraph}{0}
\ifthenelse{\boolean{prof}}{
\begin{corrige}\end{corrige}}{}

\begin{center}
\includegraphics[width=7cm]{images/moment8.pdf}
\end{center}

\subparagraph*{}
\textit{Déterminer le moment exercé par la résultante de l'action mécanique du vent sur le panneau au point $A$.}

\subsection*{Exercice 2 -- Balance romaine}
\begin{center}
\includegraphics[width=12cm]{images/balance.pdf}
\end{center}
La balance romaine représentée ci-dessus est constituée d'un contrepoids 3 coulissant le long de la tige 2 (graduée).
\subparagraph*{}
\textit{Déterminer la masse de l'objet pesé sur le crochet 4. Déterminer la force extérieure sur 1 permettant de maintenir le système à l'équilibre.}



\subsection*{Exercice 3 -- Clé à molette}
\begin{center}
\includegraphics[width=10cm]{images/moment7.pdf}
\end{center}
L'utilisateur exerce une force sur la clé au point $A$ comme indiqué sur la figure \ref{cle}. En déduire le moment de serrage de l'écrou en $B$ produit par cette force.

\subsection*{Exercice 4 -- Clé en T}
L'utilisateur exerce 2 forces égales et opposées au bout de la tige coulissante, valant $F=100$ N. Monter que dans les 4 positions l'action mécanique de serrage sur l'écroue est identique.
\begin{center}
\includegraphics[width=15cm]{images/moment2.pdf}
\end{center}


\subsection*{Exercice 5 -- Poutres}
Sur les figures suivantes, déterminer le moment exercé par le bâti sur la poutre, au point indiqué sur la figure au niveau de la liaison encastrement.


\begin{minipage}[c]{.45\linewidth}
\begin{center}
\includegraphics[width=6cm]{images/moment1.pdf}
\end{center}
\end{minipage} \hfill
\begin{minipage}[c]{.45\linewidth}
\begin{center}
\includegraphics[width=6cm]{images/moment3.pdf}
\end{center}
\end{minipage} 


\begin{minipage}[c]{.45\linewidth}
\begin{center}
\includegraphics[width=6cm]{images/moment4.pdf}
\end{center}
\end{minipage} \hfill
\begin{minipage}[c]{.45\linewidth}
\begin{center}
\includegraphics[width=6cm]{images/moment5.pdf}
\end{center}
\end{minipage} 

\begin{minipage}[c]{.45\linewidth}
\begin{center}
\includegraphics[width=6cm]{images/moment6.pdf}
\end{center}
\end{minipage} \hfill
\begin{minipage}[c]{.45\linewidth}
\begin{center}
\includegraphics[width=6cm]{images/poutre.pdf}
\end{center}
\end{minipage} 

\end{document}
