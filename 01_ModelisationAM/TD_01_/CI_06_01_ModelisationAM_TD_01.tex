\documentclass[10pt]{article}
\input{style/coursHeadings}
\input{style/programHeadings}
\input{style/macros_SII}
\input{style/macros_Titres}
\input{style/macros_Frames}

%Si le boolen xp est vrai : compilation pour xabi
%Sinon compilation Damien
\newboolean{xp}
\setboolean{xp}{true}

\newboolean{prof}
\setboolean{prof}{false}

\usepackage[%
    pdftitle={CI 06 : Stat - Modélisation des AM},
    pdfauthor={Xavier Pessoles},
    colorlinks=true,
    linkcolor=blue,
    citecolor=magenta]{hyperref}


\def\discipline{Sciences Industrielles de l'Ingénieur}
\def\xxtitre{\ifthenelse{\boolean{xp}}{
CI 06 : Étude du comportement statique des systèmes}{}}

\def\xxsoustitre{\ifthenelse{\boolean{xp}}{
Chapitre 1 -- Modélisation des Actions Mécaniques}{
Partie  -- }}

\def\xxauteur{\ifthenelse{\boolean{xp}}{
%Xavier \textsc{Pessoles}
}{}}

\def\xxpied{\ifthenelse{\boolean{xp}}{
CI 06 : Statique\\
Ch. 1 : Modélisation des AM -- TD -- Géométrie des Masses}{
\xxtitre}}

\def\xxcathegorie{\ifthenelse{\boolean{xp}}{
2013 -- 2014 \\
Xavier \textsc{Pessoles}}{}}





%---------------------------------------------------------------------------


\begin{document}

\ifthenelse{\boolean{xp}}{\input{style/enteteXP}}{\input{style/enteteDI}}

\begin{center}
\Large{\textsc{Travaux Dirigés : Modélisation des Actions Mécaniques}}
\end{center}
\begin{flushright}
\textit{Activités de Florestan Mathurin.}
\end{flushright}
\vspace{.5cm}



\section*{Exercice 1 : Barrage de la Tamise}
\ifthenelse{\boolean{prof}}{}{
\begin{minipage}[c]{.55\linewidth}

Le Thames Barrier est un barrage spectaculaire conçu pour protéger la ville de Londres des marrées exceptionnellement élevées qui peuvent remonter de la mer. Sa construction terminée en 1982 a nécessité 51 000 tonnes d'acier et 210 000 $m^3$ de béton, ce qui en fait le second barrage mobile le plus grand du monde.

\end{minipage}\hfill
\begin{minipage}[c]{.4\linewidth}
\begin{center}
\includegraphics[width=.9\textwidth]{images/fig1}
\end{center}
\end{minipage}

\vspace{.25cm}

La structure s'étend sur 520 mètres de large et est constituée de 10 portes de forme de secteur angulaire de 20 mètres de haut. Chaque porte est totalement effacée dans un berceau en béton coulé au fond de la rivière. En cas de montée des eaux, les portes pivotent en position verticale actionnées par une machine hydraulique.

\begin{center}
\includegraphics[width=.9\textwidth]{images/fig2}
\end{center}


\begin{minipage}[c]{.55\linewidth}

L'objectif est de déterminer la position du centre de gravité de la porte qui est une structure creuse en tôle épaisse et donc on donne le modèle ci contre.

\textbf{Données :}
\begin{itemize}
\item longueur porte : $L=58\;m$
\item Rayon : $R=12,4\;m$
\item épaisseur tôle : $e=0,05\;m$ (considérée négligeable devant $R$)
\item masse volumique de la porte : $\rho=7\,700 \; kg/m^3$
\item $\alpha=\pi/3$
\end{itemize}
\end{minipage}\hfill
\begin{minipage}[c]{.4\linewidth}
\begin{center}
\includegraphics[width=.9\textwidth]{images/fig3}
\end{center}
\end{minipage}
}

\subparagraph{}
\textit{Déterminer les coordonnées du centre de gravité de la porte.}
\ifthenelse{\boolean{prof}}{
\begin{corrige}

\end{corrige}}{}


\section*{Modélisation des actions mécaniques agissant sur un barrage poids}
\ifthenelse{\boolean{prof}}{}{
On s'intéresse à un barrage poids en béton de section triangulaire qui repose sur le sol et qui réalise une retenue d'eau de hauteur $h$ pour l'alimentation des voies navigables. Un barrage poids est un barrage dont le poids propre suffit à résister à la pression exercée par l'eau. Le barrage est soumis principalement à l'action mécanique de l'eau (pression hydrostatique) et à l'action mécanique de pesanteur.
\begin{center}
\includegraphics[width=.9\textwidth]{images/fig4_bis}
\end{center}

\begin{minipage}[c]{.47\linewidth}
\begin{center}
\includegraphics[width=.9\textwidth]{images/req}
\end{center}
\end{minipage} \hfill
\begin{minipage}[c]{.47\linewidth}

\textbf{Données :}
\begin{itemize}
\item $M$ : masse du barrage considéré comme un solide homogène
\item $a=20\;m$ : assise du barrage
\item $h=30\;m$ : hauteur d'eau
\item $l=80\;m$ largeur du barrage
\item $\rho_{\text{eau}} = 1\,000\;kg/m^3$ : masse volumique de l'eau.
\end{itemize}
\end{minipage} \hfill


\setcounter{subparagraph}{0}
}

\subparagraph{}
\textit{Déterminer le centre de gravité de la structure.}
\ifthenelse{\boolean{prof}}{
\begin{corrige}

\end{corrige}}{}

\subparagraph{}
\textit{Déterminer le modèle global de l'action mécanique de la pesanteur sur le barrage sous forme de torseur exprimé au centre de gravité $G$ puis au point $O$.}
\ifthenelse{\boolean{prof}}{
\begin{corrige}

\end{corrige}}{}

\subparagraph{}
\textit{Poser le modèle local puis déterminer le modèle global de l'action mécanique de l'eau sur le barrage sous forme de torseur exprimé au point pour lequel le moment résultant est nul.}
\ifthenelse{\boolean{prof}}{
\begin{corrige}

\end{corrige}}{}
\subparagraph{}
\textit{Faire les applications numériques et conclure vis-à-vis du CdCF.}
\ifthenelse{\boolean{prof}}{
\begin{corrige}

\end{corrige}}{}

\section*{Modélisation des actions mécaniques de contact sur un palier lisse}
\ifthenelse{\boolean{prof}}{}{

\begin{minipage}[c]{.55\linewidth}

On souhaite déterminer le modèle global des actions mécaniques de contact sur un palier lisse, composant technologique pour le guidage en rotation.

On donne le modèle local :
\begin{itemize}
\item les surfaces de contact sont limitées par un demi cylindre de longueur $L$ et de rayon $R$;
\item entre les surfaces de contact, la pression $p$ est uniforme sur chaque élément $dS$ situé autour du point $M$.
\end{itemize}

\end{minipage}\hfill
\begin{minipage}[c]{.4\linewidth}
\begin{center}
\includegraphics[width=.9\textwidth]{images/fig5}
\end{center}
\end{minipage}

\vspace{.25cm}
\begin{center}
\includegraphics[width=.75\textwidth]{images/fig6_bis}
\end{center}
}

\setcounter{subparagraph}{0}
\subparagraph{}
\textit{Déterminer le modèle global de l'action mécanique de l'arbre 2 sur le bâti 1 sous la forme d'un torseur exprimé au point $O$.}
\ifthenelse{\boolean{prof}}{
\begin{corrige}

Exprimons le torseur des actions mécaniques sous sa forme locale en un point $M$ : 

$$
\torseurl{d\vectf{2}{1}}{d\vectm{M}{2}{1}=\vect{0}}{M}
$$

La forme globale au point O est alors donnée par :

$$
\torseurstat{T}{2}{1} = \torseurl{\vectf{2}{1} = \int d\vectf{2}{1}}{\vectm{M}{2}{1} = \int d\vectm{M}{2}{1}= \int \vect{OM}\wedge d\vectf{2}{1}}{M}
$$

\vspace{.5cm}

\textbf{Calculons $\vectf{2}{1}$.}

$$
\vectf{2}{1} = \int d\vectf{2}{1} = \iint p \vect{-r} dS = -p \iint  \vect{r} dS
= -p \iint  \left(\cos\theta\vect{x}+\sin\theta\vect{y} \right) dS $$

$$
\vectf{2}{1}
= -p \int\limits_{-L/2}^{L/2} \int\limits_{-\pi/2}^{\pi/2}   \left(\cos\theta\vect{x}+\sin\theta\vect{y} \right) Rd\theta dz
= -p R L \int\limits_{-\pi/2}^{\pi/2}   \left(\cos\theta\vect{x}+\sin\theta\vect{y} \right) d\theta 
$$
$$
\vectf{2}{1}
= -p R L \left(\int\limits_{-\pi/2}^{\pi/2}   \cos\theta\vect{x}d\theta + \int\limits_{-\pi/2}^{\pi/2} \sin\theta\vect{y} d\theta \right)
= -p R L \left(\left[\sin\theta \right]_{-\pi/2}^{\pi/2}\vect{x}
+\left[ -\cos\theta\right]_{-\pi/2}^{\pi/2}\vect{y}
\right)
$$

$$
\vectf{2}{1}
= -p R L \left(2\vect{x}
+0\vect{y}
\right) = -2pRL\vect{x}
$$

$2RL$ est appelée surface projetée du cylindre. Elle correspond au produit du diamètre par sa longueur.

\vspace{.5cm}

\textbf{Calculons $\vectm{M}{2}{1}$.}
$$
\vectm{M}{2}{1} = \int d\vectm{M}{2}{1}= \int \vect{OM}\wedge d\vectf{2}{1}
$$

$$
\vectm{M}{2}{1} = -p \iint R\vect{r} \wedge \vect{r}dS = \vect{0}
$$

Au final, 
$$
\torseurstat{T}{2}{1} = \torseurl{\vectf{2}{1} = -2pRL\vect{x}}{\vectm{M}{2}{1} =  \vect{0}}{M}
$$



\textbf{Calculer $\vectf{2}{1}$ lorsque la pression est de la forme : $p(\theta)=p_0\cos\theta$ pour $\theta\in[-\pi/2,\pi/2]$.}

Dans ce cas : 
$$
\vectf{2}{1} = \int d\vectf{2}{1} = \iint p(\theta) \vect{-r} dS 
= - p_0R\iint\cos\theta  \left(\cos\theta\vect{x}+\sin\theta\vect{y} \right)  d\theta dz$$

$$
\vectf{2}{1} 
= - p_0 L R\int\limits_{-\pi/2}^{\pi/2}\cos\theta  \left(\cos\theta\vect{x}+\sin\theta\vect{y} \right)  d\theta$$

$$
\int\limits_{-\pi/2}^{\pi/2}\cos^2\theta  d\theta = \dfrac{\pi}{2}
\quad 
\text{et}
\quad
\int\limits_{-\pi/2}^{\pi/2}\cos\theta \sin\theta  d\theta = 0
$$
Au final :
$$
\vectf{2}{1} 
= - p_0 L R \dfrac{\pi}{2}\vect{x}$$
\end{corrige}}{}




\end{document}


