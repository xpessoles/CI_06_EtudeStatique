\documentclass[10pt]{article}
\input{style/coursHeadings}
\input{style/programHeadings}
\input{style/macros_SII}
\input{style/macros_Titres}
\input{style/macros_Frames}

%Si le boolen xp est vrai : compilation pour xabi
%Sinon compilation Damien
\newboolean{xp}
\setboolean{xp}{true}

\newboolean{prof}
\setboolean{prof}{false}

\usepackage[%
    pdftitle={CI 06 : Stat - Modélisation des AM},
    pdfauthor={Xavier Pessoles},
    colorlinks=true,
    linkcolor=blue,
    citecolor=magenta]{hyperref}


\def\discipline{Sciences Industrielles de l'Ingénieur}
\def\xxtitre{\ifthenelse{\boolean{xp}}{
CI 06 : Étude du comportement statique des systèmes}{}}

\def\xxsoustitre{\ifthenelse{\boolean{xp}}{
Chapitre 1 -- Modélisation des Actions Mécaniques}{
Partie  -- }}

\def\xxauteur{\ifthenelse{\boolean{xp}}{
Xavier \textsc{Pessoles}}{}}

\def\xxpied{\ifthenelse{\boolean{xp}}{
CI 06 : Statique\\
Ch. 1 : Modélisation des AM -- Exercice de colle}{
\xxtitre}}

\def\xxcathegorie{\ifthenelse{\boolean{xp}}{
2013 -- 2014 \\
Xavier \textsc{Pessoles}}{}}





%---------------------------------------------------------------------------


\begin{document}

\ifthenelse{\boolean{xp}}{\input{style/enteteXP}}{\input{style/enteteDI}}

\begin{center}
\Large{\textsc{Exercices de colle}}
\end{center}

\vspace{.5cm}


\subsection*{Exercice 1 : Masse et centre d'inertie}
\begin{flushright}
\textit{D'après Agati et al., Mécanique du solide, Applications industrielles, Dunod.}
\end{flushright}
\setcounter{subparagraph}{0}

\begin{minipage}[c]{.3\linewidth}
\begin{center}
\includegraphics[width=.9\textwidth]{images/disque}
\end{center}
\end{minipage}\hfill
\begin{minipage}[c]{.65\linewidth}
Soit une plaque homogène, d'épaisseur négligeable, ayant la forme d'un quart de cercle de rayon $r$. Le matériau est de masse surfacique $\mu$.

\subparagraph{}
\textit{Déterminer la masse du solide.}

\subparagraph{}
\textit{Déterminer la position du centre de gravité.}
\ifthenelse{\boolean{prof}}{
\begin{corrige}
OG = \dfrac{4\sqrt{2}}{3\pi} r 
\end{corrige}
}{}


\subparagraph{}
\textit{Répondre aux même questions pour un huitième de boule.}

\end{minipage}


\subsection*{Détermination des actions mécaniques}
\setcounter{subparagraph}{0}
\begin{minipage}[c]{.3\linewidth}
\begin{center}
\includegraphics[width=.9\textwidth]{images/bateau}
\end{center}
\end{minipage} \hfill
\begin{minipage}[c]{.67\linewidth}
Deux rotors verticaux de 2 mètres de diamètre et de 8 mètres de hauteur, en rotation autour de leur axe à la vitesse de 200 tr/min sont utilisés pour la propulsion d'un navire. 

Le vecteur vitesse relative $V_0 \vect{x}$ du vent par rapport au navire est de 40 km/h et il est perpendiculaire au navire. 

La force propulsive qui entraine le navire est due à l'\textit{effet Magnus} : lorsqu'on place dans un courant d'air un cylindre animé d'un mouvement de rotation autour de son axe, le cylindre met en mouvement de rotation autour de son axe, le cylindre met en mouvement par viscosité les filets fluides qui le contournent, et qui engendrent alors une action mécanique sur le cylindre, représentée par une force perpendiculaire à la direction du courant d'air dont le sens dépend du sens de rotation du cylindre. 
%Cet effet explique aussi la courbe que présente la trajectoire d'une balle de tennis coupée. 

On montre en aérodynamique que dans un tel écoulement la répartition des pressions sur le cylindre est :
$$
\vect{p}(M)=-\dfrac{1}{2}\rho V_0^2 \left[ 1-\left( \dfrac{\omega r}{V_0}-2\sin\alpha\right)^2\right]\vect{n}
$$

\end{minipage}


On note : 
\begin{itemize}
\item $\rho=1,225\; kg/m^3$ : masse volumique de l'air; 
\item $V_0$ : vitesse du vent suivant $(O,\vect{x})$;
\item $\omega$ : vitesse de rotation du cylindre autour de l'axe $(O,\vect{z})$;
\item $r$ : rayon du cylindre;
\item $\vect{n}$ : vecteur unitaire normal à la surface latérale du cylindre, orienté vers l'extérieur du cylindre;
\item $\alpha = \left( \vect{x};\vect{n}\right)$.
\end{itemize}

$V_0$ et $\omega$ sont algébriques.

\subparagraph{}
\textit{Montrer que la résultante générale des pressions aérodynamiques est par unité de longueur du cylindre :
$$
\vect{F}=-2\pi \rho  V_0 \omega r^2 \vect{y}
$$}
\ifthenelse{\boolean{prof}}{
\begin{corrige}

\end{corrige}}{}

\subparagraph{}
\textit{En dédire la forme propulsive théorique qui entraîne le navire.}
\ifthenelse{\boolean{prof}}{
\begin{corrige}

\end{corrige}}{}

\subsection*{Exercice 3 : Frein à disque}
\setcounter{subparagraph}{0}
\begin{flushright}
\textit{D'après ressources de Florestan Mathurin.}
\end{flushright}
\begin{minipage}[c]{.6\linewidth}
Pour ralentir ou immobiliser un système en mouvement, il est nécessaire de disposer d'un système de freinage. Le frein à disque est une solution technique permettant de réaliser le freinage d'un véhicule (moto, automobile...). Il est constitué d'un disque fixé sur le moyeu ou la jante de la roue (disque ayant le même mouvement de rotation que la roue) ainsi que des plaquettes venant frotter de chaque côté du disque. Les plaquettes sont maintenues dans un étrier lié au véhicule. Un ou plusieurs mécanismes poussent sur les plaquettes, le plus souvent des pistons hydrauliques, les plaquettes viennent serrer fortement le disque. La force de frottement entre les plaquettes et le disque crée un couple de freinage diminuant voire immobilisant la rotation de la roue. 
\end{minipage} \hfill
\begin{minipage}[c]{.35\linewidth}
\begin{center}
\includegraphics[width=\textwidth]{images/frein1}
\end{center}
\end{minipage}

\begin{minipage}[c]{.4\linewidth}
\begin{center}
\includegraphics[width=.95\textwidth]{images/frein2}
\end{center}
\end{minipage} \hfill
\begin{minipage}[c]{.57\linewidth}
L'appui sur la pédale de frein entraîne une augmentation de pression qui se retrouve au niveau des pistons. Ceux-ci poussent les plaquettes contre le disque. Un effort normal au disque apparaît alors. Par le frottement des plaquettes sur le disque, les efforts tangentiels viennent créer le couple de freinage. 

On utilise le modèle suivant pour déterminer la relation entre l'effort presseur N exercé sur les plaquettes et le couple de freinage C dans un frein à disque. 
\end{minipage}



\begin{minipage}[c]{.3\linewidth}
\begin{center}
\includegraphics[width=\textwidth]{images/frein3}
\end{center}
\end{minipage} \hfill
\begin{minipage}[c]{.62\linewidth}
La plaquette est modélisée par une portion de couronne de rayons $r_1$ et $r_2$ et d'angle $2\alpha$ considérée en liaison glissière avec le bâti 0 suivant l’axe $(O,\vect{z})$. 

On note $M$ un point de la plaquette défini en coordonnées polaires par $\vect{OM}=r\vect{e_r}$
avec $\theta=\left(\vect{x},\vect{e_r}\right)$.
On note $p$ la pression exercée par les plaquettes sur le disque d’épaisseur 2e. On suppose que la pression $p$ est constante.  

On note $f$ (constante) le facteur de frottement entre les plaquettes et le disque.  On note N la résultante sur l’axe $\vect{z}$ de l'action mécanique exercée par un piston sur une plaquette. 
\end{minipage}



\subparagraph{}
\textit{En précisant le théorème utilisé et le système isolé, déterminer une relation entre $p$ et $N$ en fonction de $r_1$ , $r_2$  et $\alpha$. }

\subparagraph{}
\textit{Sachant que lors du freinage il y a glissement et que $\vecto{1}{0}=\dot{\theta}_{10}\cdot \vect{z}$ avec $\dot{\theta}_{10}>0$, déterminer l'action mécanique élémentaire de 2 sur 1 et le modèle local de l’action mécanique de 2 sur 1 au point $O$. }

\subparagraph{}
\textit{Déterminer le torseur de l'action mécanique globale de 2 sur 1 au point O. En déduire le couple de freinage en fonction de $\alpha$, $N$, $f$, $r_1$ et $r_2$ sachant qu'il y a deux surfaces de frottement (une plaquette de part et d'autre du disque). }
\end{document}


