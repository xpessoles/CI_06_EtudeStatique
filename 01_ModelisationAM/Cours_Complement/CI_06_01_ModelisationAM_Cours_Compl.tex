\documentclass[10pt]{article}
\input{style/coursHeadings}
\input{style/programHeadings}
\input{style/macros_SII}
\input{style/macros_Titres}
\input{style/macros_Frames}

%Si le boolen xp est vrai : compilation pour xabi
%Sinon compilation Damien
\newboolean{xp}
\setboolean{xp}{true}

\newboolean{prof}
\setboolean{prof}{true}

\usepackage[%
    pdftitle={CI 06 : Stat - Modélisation des AM},
    pdfauthor={Xavier Pessoles},
    colorlinks=true,
    linkcolor=blue,
    citecolor=magenta]{hyperref}


\def\discipline{Sciences Industrielles de l'Ingénieur}
\def\xxtitre{\ifthenelse{\boolean{xp}}{
CI 06 : Étude du comportement statique des systèmes}{}}

\def\xxsoustitre{\ifthenelse{\boolean{xp}}{
Chapitre 1 -- Modélisation des Actions Mécaniques}{
Partie  -- }}

\def\xxauteur{\ifthenelse{\boolean{xp}}{
Xavier \textsc{Pessoles}}{}}

\def\xxpied{\ifthenelse{\boolean{xp}}{
CI 06 : Statique\\
Ch. 1 : Modélisation des AM -- Cours}{
\xxtitre}}

\def\xxcathegorie{\ifthenelse{\boolean{xp}}{
2013 -- 2014 \\
Xavier \textsc{Pessoles}}{}}





%---------------------------------------------------------------------------


\begin{document}

\ifthenelse{\boolean{xp}}{\input{style/enteteXP}}{\input{style/enteteDI}}


\begin{center}
\large{\textsc{Résistance au roulement et au pivotement}}
\end{center}

\vspace{.5cm}


\begin{savoir}
\textsc{Savoirs :}
\begin{itemize}
\item Mod-C15 : Modélisation des actions mécaniques
\begin{itemize}
\item Mod-C15.1 : Modèle local (densité surfacique, linéique et volumique d’effort) :
\begin{itemize}
%\item contact parfait ;
%\item modélisation du frottement sec -- Lois de Coulomb ;
\item modélisation de résistance au roulement;
\item modélisation de résistance au pivotement;
\end{itemize}
%\item Mod-C15.2 : Modèle global (torseur d’action mécanique)
%\item Mod-C15.3 : Modèle global du frottement visqueux.
\end{itemize}
\begin{itemize}
\item Mod-C15-S3 : Écrire le modèle global de l’action de la pesanteur, du frottement fluide, de la résistance au roulement et du pivotement.

.
\end{itemize}
\end{itemize}
\end{savoir}

\setlength{\parskip}{0ex plus 0.2ex minus 0ex}
 \renewcommand{\contentsname}{}
 \renewcommand{\baselinestretch}{1}

\tableofcontents

 \renewcommand{\baselinestretch}{1.2}
\setlength{\parskip}{2ex plus 0.5ex minus 0.2ex}

% \vspace{1cm}
\textit{Ce document est en évolution permanente. Merci de signaler toutes
erreurs ou coquilles.}


\section{Résistance au roulement et au pivotement}

\subsection{Définitions}

On a vue dans le cas général d'un contact surfacique entre deux solides, le torseur des actions mécaniques pouvaient s'exprimer ainsi.

$$
\torseurstat{T}{S_2}{S_1}=
\torseurl{%
\vect{R_{(S_2\rightarrow S_1)}} 
= \iint\limits_{\mathcal{S}} f(M) \vect{u(M)} \text{d}\mathcal{S}}{%
\vectm{P}{S_2}{S_1} = \iint\limits_{\mathcal{S}}\vect{PM}\wedge d\vectf{ext}{S}}{M}
$$

Avec :
\begin{itemize}
\item $f(M) \vect{u(M)} = p_{21}(M)\vect{n_{21}}+\vect{\tau_{21}}(M)$;
\item $p_{21}(M)$ : pression de contact au point $M$ (en $N/m^2$);
\item $\vect{\tau_{21}}(M)$ : la projection tangentielle de la densité surfacique (norme en $N/m^2$).
\end{itemize}


\begin{defi}
\textbf{Résistance au roulement}

On définit l'effort normal au contact :
$$
\vect{N\left(S_2\rightarrow S_1\right)} = \iint\limits_{\mathcal{S}}  p_{21}(M)\vect{n_{21}}\text{d}\mathcal{S}
$$

On définit le moment résistant au roulement :
$$
\vect{\mathcal{M}_r\left(P,S_2\rightarrow S_1\right)} = \iint\limits_{\mathcal{S}}  \vect{PM} \wedge  p_{21}(M)\vect{n_{21}}\text{d}\mathcal{S}
$$

$P$ étant un point de l'axe de rotation du roulement, on a :

$$
||\vect{\mathcal{M}_r\left(P,S_2\rightarrow S_1\right)}|| \leq \mu_r || \vect{N\left(S_2\rightarrow S_1\right)}  ||
$$

Il y a égalité entre les deux termes lorsqu'on est à la limite du glissement, ou lorsque $\vecto{S_1}{S_2}_r\neq 0$.

On note $\mu_r$ (en mètres) le coefficient de résistance au roulement. 
\end{defi}



\begin{defi}
\textbf{Résistance au pivotement}


On définit le moment résistant au pivotement :
$$
\vect{\mathcal{M}_p\left(P,S_2\rightarrow S_1\right)} = \iint\limits_{\mathcal{S}}  \vect{PM} \wedge \vect{\tau_{21}}(M) \text{d}\mathcal{S}
$$

$P$ étant un point de l'axe de rotation de pivotement, on a :

$$
||\vect{\mathcal{M}_p\left(P,S_2\rightarrow S_1\right)}|| \leq \mu_p || \vect{N\left(S_2\rightarrow S_1\right)}  ||
$$

Il y a égalité entre les deux termes lorsqu'on est à la limite du glissement, ou lorsque $\vecto{S_2}{S_1}_p\neq 0$.

On note $\mu_p$ (en mètres) le coefficient de résistance au pivotement. 
\end{defi}

\subsection{Résistance au roulement}
\begin{hypo}
\begin{itemize}
\item La roue 2 roule sans glisser sur le sol 1;
\item la roue avance en ligne droite; 
\item il n'y a pas d'effort latéral. 
\end{itemize}
\end{hypo}

\begin{minipage}[c]{.45\linewidth}
\begin{center}
\includegraphics[width=.6\textwidth]{images/roule_01}

\textit{Contact théorique ponctuel en $I$ situé sous le point $M$}
\end{center}
\end{minipage} \hfill
\begin{minipage}[c]{.45\linewidth}
\begin{center}
\includegraphics[width=.6\textwidth]{images/roule_02}

\textit{Contact réel}
\end{center}
\end{minipage}

Écrivons le torseur cinématique $\torseurcin{V}{2}{1}$ et le torseur des actions mécaniques $\torseurstat{T}{1}{2}$ en faisant l'hypothèse de problème plan :
$$
\torseurcin{V}{2}{1} 
=\torseurcol{0}{0}{\omega_{21}}{0}{0}{0}{I}
\quad 
\torseurstat{T}{1}{2} 
=\torseurcol{X}{Y}{0}{0}{0}{N}{I}
$$

Avec $N = \mu_r Y$. 

\begin{rem}
Quelques valeurs de $\mu_r$ :
\begin{itemize}
\item $\mu_r$ varie de 1 à 2 mm pour une roue de wagon;
\item $\mu_r$ varie de 6 à 8 mm pour une roue d'automobile;
\item $\mu_r \simeq 0,497 \; mm$ pour les pneus Michelin utilisés à l'éco-marathon Shell.
\end{itemize}
\end{rem}

\subsubsection{Recherchons le point d'application de l'effort de contact}

\begin{minipage}[c]{.65\linewidth}
$\torseurstat{T}{1}{2}$ est un glisseur car son invariant scalaire est nul. Cherchons donc le point central $K$ situé dans le plan du sol :
$\vectm{I}{1}{2} = \vectm{K}{1}{2} + \vect{IK}\wedge \vectf{1}{2}$. 
En posant $\vect{IK} = x\vect{x} + y\vect{y}$, on obtient $\mu_r Y = x Y$ et $z=0$.

La forge agit donc en avant de point de contact théorique. 

La représentation graphique du torseur se résume au seul vecteur résultante car c'est un glisseur place en un point central.

L'effort est dans le cône d'adhérence lorsqu'il y a roulement sans glissement. On est sur le cône à la limite de l'adhérence. 
\end{minipage} \hfill
\begin{minipage}[c]{.3\linewidth}
\begin{center}
\includegraphics[width=.95\textwidth]{images/roule_03}

\textit{Contact réel}
\end{center}
\end{minipage}


\subsection{Application}
\subsubsection{Énoncé}
\begin{center}
\includegraphics[width=.6\textwidth]{images/roule_04}
\end{center}


Le véhicule modélisé ci-dessus comporte deux roues et possède un plan de symétrie dans lequel il est représenté. Les liaisons en A et en B sont parfaites. Seul le poids du châssis (comprenant le moteur) situé au milieu des roues est supposé non nul et vaut $\{P\}$. Il se déplace vers la gauche grâce à un moteur entraînant la roue 3 (le moteur fournit à 3 un couple d'axe $\vect{z}$).

Il existe, au niveau du contact entre les deux roues et le sol, un coefficient de roulement $\mu_r$. Les roues ont un rayon R. La distance AB vaut a.

On cherche le couple $C_3$ au niveau de la roue motrice que doit développer le moteur pour maintenir le véhicule à vitesse constante sur le sol 1 qui est horizontal. On pourra donc appliquer le principe fondamental de la statique.

\subsubsection{Résolution}

On fait l'hypothèse de problème plan. Ainsi, $\torseurstat{T}{i}{j}=\torseurcol{X}{Y}{0}{0}{0}{N}{P,\mathcal{B}}$.

\textbf{On isole la roue non motrice 2 et on fait le bilan des actions mécaniques extérieures :}
\begin{itemize}
\item le roulement de la roue 2 sur le sol entraîne l'existence d'une force dont le point d'application $K$ est <<en avant>> de la roue, dans le sens de la marche;
\item l'effort du châssis 4 sur la roue 2 en $A$
\end{itemize}

Le solide 2 étant soumis à 2 glisseurs, l'application du PFS implique que le support des efforts est la droite $(AK)$.

On montre alors que $\torseurstat{T}{1}{2}=\torseurcol{Y_K\tan\alpha}{Y_K}{0}{0}{0}{N}{K,\mathcal{B}}$

\textbf{On isole l'ensemble \{2,3,4\} et on fait le bilan des actions mécaniques extérieures :}
\begin{itemize}
\item le roulement de la roue 3 sur le sol entraîne l'existence d'une force dont le point d'application $L$ est <<en avant>> de la roue, dans le sens de la marche :
$$\torseurstat{T}{1}{3}=\torseurcol{X_L}{Y_L}{0}{0}{0}{0}{K,\mathcal{L}}$$
\item effort du sol sur la roue en $K$ :
$$\torseurstat{T}{1}{2}=\torseurcol{Y_K\tan\alpha}{Y_K}{0}{0}{0}{N}{K,\mathcal{B}}$$
\item pesanteur en $G$ :
$$\torseurstat{T}{\text{pes}}{4}=\torseurcol{0}{-P}{0}{0}{0}{0}{G,\mathcal{B}}$$
\end{itemize}

On applique le PFS au système isolé en L :

$$\torseurstat{T}{1}{3} + \torseurstat{T}{1}{2} + \torseurstat{T}{\text{pes}}{4} =\{0\}$$

La résolution du système conduit à :
$$
\left\{
\begin{array}{l}
Y_K\tan\alpha + X_L = 0 \\
Y_K-P + Y_L = 0 \\
-aY + \left( \dfrac{a}{2}-\mu_r\right) P = 0
\end{array}
\right.
\Longrightarrow
\left\{
\begin{array}{l}
Y_K= \dfrac{\left( \dfrac{a}{2} - \mu_r\right) P}{a} \\
X_L = - \left( \dfrac{a}{2} - \mu_r \right) \dfrac{\mu_r P}{aR} \\
Y_L = = 0
\end{array}
\right.
$$

\textbf{On isole la roue motrice 3 et on fait le bilan des actions mécaniques extérieures.}
La roue motrice est soumise aux actions du sol, de la pivot en $B$ et d'un couple moteur. 
En écrivant le théorème du moment statique appliqué à 3 en $B$, on obtient 
$$
C_m = \mu_r P
$$

Ainsi le couple à fournir par le moteur dépend du coefficient de roulement est du poids du véhicule.

\begin{thebibliography}{2}
\bibitem{beyent}{Patrick Beynet et Al., Formulaire Maths, Physique, Chimie, SII. Éditions Ellipses.}
\bibitem{pupier}{JP Pupier, Résistance au roulement. Cours de PTSI. Lycée Rouvière.}
\end{thebibliography}


\end{document}


