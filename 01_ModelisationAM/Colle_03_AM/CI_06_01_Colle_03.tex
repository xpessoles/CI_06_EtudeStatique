\documentclass[10pt]{article}
\input{style/coursHeadings}
\input{style/programHeadings}
\input{style/macros_SII}
\input{style/macros_Titres}
\input{style/macros_Frames}

%Si le boolen xp est vrai : compilation pour xabi
%Sinon compilation Damien
\newboolean{xp}
\setboolean{xp}{true}

\newboolean{prof}
\setboolean{prof}{false}

\usepackage[%
    pdftitle={CI 06 : Stat - Modélisation des AM},
    pdfauthor={Xavier Pessoles},
    colorlinks=true,
    linkcolor=blue,
    citecolor=magenta]{hyperref}


\def\discipline{Sciences Industrielles de l'Ingénieur}
\def\xxtitre{\ifthenelse{\boolean{xp}}{
CI 06 : Étude du comportement statique des systèmes}{}}

\def\xxsoustitre{\ifthenelse{\boolean{xp}}{
Chapitre 1 -- Modélisation des Actions Mécaniques}{
Partie  -- }}

\def\xxauteur{\ifthenelse{\boolean{xp}}{
Xavier \textsc{Pessoles}}{}}

\def\xxpied{\ifthenelse{\boolean{xp}}{
CI 06 : Statique\\
Ch. 1 : Modélisation des AM -- Exercice de colle}{
\xxtitre}}

\def\xxcathegorie{\ifthenelse{\boolean{xp}}{
2013 -- 2014 \\
Xavier \textsc{Pessoles}}{}}





%---------------------------------------------------------------------------


\begin{document}

\ifthenelse{\boolean{xp}}{\input{style/enteteXP}}{\input{style/enteteDI}}

\begin{center}
\Large{\textsc{Exercices de colle}}
\end{center}

\vspace{.5cm}


\subsection*{Exercice 1 : Masse et centre d'inertie}
\begin{flushright}
\textit{D'après Agati et al., Mécanique du solide, Applications industrielles, Dunod.}
\end{flushright}
\setcounter{subparagraph}{0}

\begin{minipage}[c]{.3\linewidth}
\begin{center}
\includegraphics[width=.95\textwidth]{images/plaque}
\end{center}
\end{minipage}\hfill
\begin{minipage}[c]{.65\linewidth}
Soit une plaque homogène, d'épaisseur négligeable, ayant la forme ci-contre. Le matériau est de masse surfacique $\mu$.

\subparagraph{}
\textit{Déterminer la masse du solide.}

\subparagraph{}
\textit{Déterminer la position du centre de gravité.}
\ifthenelse{\boolean{prof}}{
\begin{corrige}

\end{corrige}
}{}


\end{minipage}


\subsection*{Exercice 2 : Détermination des actions mécaniques}
\setcounter{subparagraph}{0}
\begin{minipage}[c]{.3\linewidth}
\begin{center}
%\includegraphics[width=\textwidth]{images/flotteur_1}
\end{center}
\end{minipage} \hfill
\begin{minipage}[c]{.67\linewidth}
Un flotteur de carburateur peut être assimilé à un tronc de cône de révolution. 

En chaque point $M$ de la surface immergée du flotteur, l'essence exerce une action mécanique définie par la densité surfacique : 
$$
\vect{f_M} = -\rho  \left(h-z\right) \vect{n}
$$
On note : 
\begin{itemize}
\item $\rho$ : masse volumique de l'essence; 
\item $g$ : accélération de la pesanteur;
\item $\vect{n}$ : vecteur unitaire normal à la surface au point $M$, orienté vers l'extérieur du flotteur;
\item $z$ : abscisse du point $M$ sur l'axe $\left(O,\vect{z}\right)$.
\end{itemize}
Dans le but de lester le flotteur, il est nécessaire de connaître l'action mécanique exercée par l'essence. 
\end{minipage}

\begin{center}
\includegraphics[width=.5\textwidth]{images/flotteur_1}
\end{center}

\vspace{.25cm}

\subparagraph{}
\textit{Montrer que le torseur des actions mécaniques des forces de pression exercées par l'essence sur le flotteur s'écrit au point $O$:
$$
\torseurstat{T}{\text{essence}}{\text{flotteur}}=\torseurl{\vect{R}}{\vect{0}}{O}
$$
avec $\vect{R}=\pi \rho h g \left[ r^2 h \tan \alpha \left(r+\dfrac{h}{3}\tan \alpha \right)\right]$.}
\ifthenelse{\boolean{prof}}{
\begin{corrige}

\end{corrige}}{}

\end{document}


