\documentclass[10pt]{article}
\input{style/coursHeadings}
\input{style/programHeadings}
\input{style/macros_SII}
\input{style/macros_Titres}
\input{style/macros_Frames}

%Si le boolen xp est vrai : compilation pour xabi
%Sinon compilation Damien
\newboolean{xp}
\setboolean{xp}{true}

\newboolean{prof}
\setboolean{prof}{false}

\usepackage[%
    pdftitle={CI 06 : Stat - Modélisation des AM},
    pdfauthor={Xavier Pessoles},
    colorlinks=true,
    linkcolor=blue,
    citecolor=magenta]{hyperref}


\def\discipline{Sciences Industrielles de l'Ingénieur}
\def\xxtitre{\ifthenelse{\boolean{xp}}{
CI 06 : Étude du comportement statique des systèmes}{}}

\def\xxsoustitre{\ifthenelse{\boolean{xp}}{
Chapitre 1 -- Modélisation des Actions Mécaniques}{
Partie  -- }}

\def\xxauteur{\ifthenelse{\boolean{xp}}{
Xavier \textsc{Pessoles}}{}}


\def\xxpied{\ifthenelse{\boolean{xp}}{
CI 06 : Statique\\
Ch. 1 : Modélisation des AM -- TD -- Géométrie des Masses}{
\xxtitre}}


\def\xxcathegorie{\ifthenelse{\boolean{xp}}{
2013 -- 2014 \\
Xavier \textsc{Pessoles}}{}}



%---------------------------------------------------------------------------


\begin{document}

\ifthenelse{\boolean{xp}}{\input{style/enteteXP}}{\input{style/enteteDI}}

\begin{center}
\Large{\textsc{Exercices d'application : Géométrie des Masses}}
\end{center}

\vspace{.5cm}

\ifthenelse{\boolean{prof}}{}{
\begin{defi}
\textbf{Masse d'un solide}

Soit un solide de masse volumique $\mu$ et de volume $\mathcal{V}$. Sa masse $m$ est définie par : 
$$
m=\int\limits_{\mathcal{V}}\mu d\mathcal{V}
$$

\end{defi}

\begin{defi}
\textbf{Centre d'inertie d'un solide}

Soit un point $M$ quelconque et un solide $S$ de masse $m$. Soit $P$ un point appartenant à $S$. Le centre d'inertie (ou centre de gravité ou centre de masse) est le point $G$ défini par : 
$$
\vect{MG}=\dfrac{1}{m} \int\vect{MP} dm
$$

\end{defi}


\begin{defi}
\textbf{Centre d'inertie d'un système matériel}

Soient un système matériel $E$ de masse $m$, composé de $n$ solides $S_i$ de masse $m_i$ et de centre d'inertie $G_i$. Le centre d'inertie d'inertie $G$ de $E$ est défini par : 
$$
\vect{MG} =\dfrac{1}{m}\sum\limits_{i=1}^{n} m_i \vect{MG_i}
$$
\end{defi}}


\subsection*{Exercice 1 -- Parallélépipède rectangle}
\begin{minipage}[c]{.3\linewidth}
\begin{center}
\includegraphics[width=.95\textwidth]{images/parallelepipede}
\end{center}
\end{minipage} \hfill
\begin{minipage}[c]{.65\linewidth}
Soit un parallélépipède rectangle en matériau de masse volumique $\mu$.
\subparagraph{}
\textit{Déterminer la masse du solide.}
\subparagraph{}
\textit{Déterminer la position du centre de gravité.}
\end{minipage}
\ifthenelse{\boolean{prof}}{
\begin{corrige}
On a :
$$
m=\int\limits_{\mathcal{V}}\mu \text{d}\mathcal{V}
$$

Le solide étant un parallélépipède rectangle, le système de coordonnée le mieux adapté est le système cartésien. 

Un élément infinitésimal de volume peut donc s'écrire ainsi :

$$
 \text{d}\mathcal{V} = \text{d}x \text{d}y \text{d}z
$$

$x$ variera de $0$ à $a$, $y$ variera de $0$ à $b$, $z$ variera de $0$ à $c$.

  On a donc :
$$
m
=\iiint\limits_{\mathcal{V}}\mu \text{d}\mathcal{V} 
= \int\limits_{0}^{c}\int\limits_{0}^{b}\int\limits_{0}^{a}\mu \text{d}x\text{d}y\text{d}z
= \mu \int\limits_{0}^{a}\text{d}x \int\limits_{0}^{b}\text{d}y\int\limits_{0}^{c}\text{d}z
= \mu \left[ x\right]_0^a \cdot \left[ y\right]_0^b \cdot \left[ z\right]_0^c
$$

D'où 
$$
m = a b c \mu 
$$

La position du centre d'inertie $G$ est définie par :
$$
\vect{OG}
=\dfrac{1}{m} \int\vect{OP} \mathrm{d}m
$$

La position d'un point quelconque $P$ est définie par $\vect{OP}=x\vect{i}+y\vect{j}+z\vect{k}$. On calcule donc :

$$
\vect{OG}
= \dfrac{1}{m} \int\vect{OP} \mathrm{d}m 
= \dfrac{1}{m} \iiint \left( x\vect{i}+y\vect{j}+z\vect{k}\right) \mathrm{d}m
= \dfrac{1}{m}  \iiint x\vect{i} \mathrm{d}m + \dfrac{1}{m} \iiint y\vect{j} \mathrm{d}m + \dfrac{1}{m} \iiint z\vect{k} \mathrm{d}m  
$$

En séparant les calculs :
\begin{eqnarray*}
\iiint x\vect{i} \mathrm{d}m 
&=& \int_{0}^{c}\int_{0}^{b}\int_{0}^{a} x\vect{i} \mu \mathrm{d}x \mathrm{d}y \mathrm{d}z
= \mu \vect{i} \left( \int_{0}^{c} \mathrm{d}z \cdot \int_{0}^{b} \mathrm{d}y \cdot \int_{0}^{a} x\mathrm{d}x \right) \\
&= & \mu \left(\left[ \dfrac{x^2}{2}\right]_0^a \cdot \left[ y\right]_0^b \cdot \left[ z\right]_0^c \right)\vect{i}
= \dfrac{a^2}{2} \cdot b \cdot c\vect{i} = \mu abc \dfrac{a}{2} \vect{i}
\end{eqnarray*}

En faisant de même pour les autres termes, on a : 
$$
\vect{OG}
= \dfrac{\mu a b c }{m}\left( \dfrac{a}{2}\vect{i}+\dfrac{b}{2}\vect{j}+\dfrac{c}{2}\vect{k}\right)
= \dfrac{a}{2}\vect{i}+\dfrac{b}{2}\vect{j}+\dfrac{c}{2}\vect{k}
$$
\end{corrige}}{}


\subsection*{Exercice 2 -- Cylindre}
\setcounter{subparagraph}{0}
\begin{minipage}[c]{.3\linewidth}
\begin{center}
\includegraphics[width=.95\textwidth]{images/cylindre}
\end{center}
\end{minipage} \hfill
\begin{minipage}[c]{.65\linewidth}
Soit un volume cylindrique de masse volumique $\mu$.
\subparagraph{}
\textit{Déterminer la masse du solide.}
\subparagraph{}
\textit{Déterminer la position du centre de gravité.}
\end{minipage}

\ifthenelse{\boolean{prof}}{
\begin{corrige}

\end{corrige}}{}

\subsection*{Exercice 3 -- Boule}
\setcounter{subparagraph}{0}
\begin{minipage}[c]{.3\linewidth}
\begin{center}
\includegraphics[width=.95\textwidth]{images/sphere_2}
\end{center}
\end{minipage} \hfill
\begin{minipage}[c]{.65\linewidth}
Soit une boule de masse volumique $\mu$.
\subparagraph{}
\textit{Déterminer la masse du solide.}
\subparagraph{}
\textit{Déterminer la position du centre de gravité.}
\end{minipage}
\ifthenelse{\boolean{prof}}{
\begin{corrige}
\begin{center}
\includegraphics[width=.35\textwidth]{images/coord_spheriques}
\includegraphics[width=.25\textwidth]{images/sphere}
\end{center}
\end{corrige}}{}

\subsection*{Exercice 4 -- Portion de cylindre}
\setcounter{subparagraph}{0}
\begin{minipage}[c]{.35\linewidth}
\begin{center}
\includegraphics[width=\textwidth]{images/portioncylindre}
\end{center}
\end{minipage} \hfill
\begin{minipage}[c]{.65\linewidth}
Soit une portion cylindrique de masse volumique $\mu$ et de secteur angulaire 2$\theta$.
\subparagraph{}
\textit{Déterminer la masse du solide.}
\subparagraph{}
\textit{Déterminer la position du centre de gravité.}
\end{minipage}
\ifthenelse{\boolean{prof}}{
\begin{corrige}

\end{corrige}}{}

\subsection*{Exercice 5 -- Portion de cylindre}
\setcounter{subparagraph}{0}
\begin{minipage}[c]{.35\linewidth}
\begin{center}
\includegraphics[width=\textwidth]{images/cylindre}
\end{center}
\end{minipage} \hfill
\begin{minipage}[c]{.65\linewidth}
Soit une portion cylindrique de masse volumique $\mu$, de secteur angulaire 2$\theta$ et d'épaisseur $e$. 
\subparagraph{}
\textit{Déterminer la masse du solide.}
\subparagraph{}
\textit{Déterminer la position du centre de gravité.}
\end{minipage}
\ifthenelse{\boolean{prof}}{
\begin{corrige}

\end{corrige}}{}

\subsection*{Exercice 6 -- Prisme}
\setcounter{subparagraph}{0}
\begin{minipage}[c]{.3\linewidth}
\begin{center}
\includegraphics[width=\textwidth]{images/prisme}
\end{center}
\end{minipage} \hfill
\begin{minipage}[c]{.65\linewidth}
Soit un prisme de masse volumique $\mu$ et de profondeur $L$. 
\subparagraph{}
\textit{Déterminer la masse du solide.}
\subparagraph{}
\textit{Déterminer la position du centre de gravité.}
\end{minipage}
\ifthenelse{\boolean{prof}}{
\begin{corrige}

\end{corrige}}{}

\end{document}


