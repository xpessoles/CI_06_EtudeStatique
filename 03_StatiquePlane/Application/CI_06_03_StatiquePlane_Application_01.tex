\documentclass[10pt]{article}
\input{style/coursHeadings}
\input{style/programHeadings}
\input{style/macros_SII}
\input{style/macros_Titres}
\input{style/macros_Frames}

%Si le boolen xp est vrai : compilation pour xabi
%Sinon compilation Damien
\newboolean{xp}
\setboolean{xp}{true}

\newboolean{prof}
\setboolean{prof}{true}

\usepackage[%
    pdftitle={CI 06 : Stat - Problèmes plans},
    pdfauthor={Xavier Pessoles},
    colorlinks=true,
    linkcolor=blue,
    citecolor=magenta]{hyperref}


\def\discipline{Sciences Industrielles de l'Ingénieur}
\def\xxtitre{\ifthenelse{\boolean{xp}}{
CI 6 : Étude du comportement statique des systèmes}{}}

\def\xxsoustitre{\ifthenelse{\boolean{xp}}{
Chapitre 3 -- Résolution des problèmes de statique plane}{
Partie  -- }}

\def\xxauteur{\ifthenelse{\boolean{xp}}{
Xavier \textsc{Pessoles}
}{}}

\def\xxpied{\ifthenelse{\boolean{xp}}{
CI 6 : Statique\\
Ch. 3 : Statique plane -- Application}{
\xxtitre}}

\def\xxcathegorie{\ifthenelse{\boolean{xp}}{
2013 -- 2014 \\
Xavier \textsc{Pessoles}}{}}

%---------------------------------------------------------------------------

\begin{document}

\ifthenelse{\boolean{xp}}{\input{style/enteteXP}}{\input{style/enteteDI}}

\begin{center}
\Large{\textsc{Exercices d'applications}}
\end{center}

\begin{flushright}
\textit{D'après Guide de Mécanique, Jean-Louis Fanchon.}
\end{flushright}

\begin{rem}
Même si l'objectif est ici de présenter les méthodes graphiques de résolution des problèmes de statiques, la \textbf{méthode devra} néanmoins apparaître clairement. 

On devra en particulier préciser le système isolé. 

L'action mécanique de 1 sur 2 au point $P$ sera notée $\vect{P_{12}}$.

Lors de la réalisation du bilan des actions mécaniques, pour chaque effort, il faudra préciser, dans la mesure du possible :
\begin{itemize}
\item le point d'application;
\item la norme de l'effort;
\item la direction et le sens.
\end{itemize}
\end{rem}


\subsection*{Ensemble matériel soumis à 2 actions mécaniques}
\begin{resultat}
Soit un solide ou un ensemble matériels soumis à deux actions mécaniques. 

D'après le PFS, ces forces sont :
\begin{itemize}
\item de même norme;
\item de même direction, la direction passant par le point d'application des deux forces;
\item de sens opposé.
\end{itemize}

En conséquence, dans un système, lorsqu'un solide est soumis à deux actions mécaniques, on peut directement déduire la direction des actions mécaniques. 
\end{resultat}

\begin{minipage}[c]{.6\linewidth}
On considère l'échelle d'un camion de pompier ci-contre. Les solides 1 et 2 sont considérés comme encastrés. 

\subparagraph{}
\textit{On isole l'ensemble 4+5. Réaliser le BAME. Comment se traduit l'application du PFS ?}

\end{minipage} \hfill
\begin{minipage}[c]{.37\linewidth}
\begin{center}
\includegraphics[width=.95\textwidth]{images/echelle}
\end{center}
\end{minipage} 

\subsection*{Ensemble matériel soumis à 3 actions mécaniques non parallèles}

\begin{resultat}
Soit un solide ou un ensemble matériels soumis à deux actions mécaniques. 

D'après le PFS :
\begin{itemize}
\item les supports des 3 forces sont coplanaires;
\item les supports des 3 forces sont concourantes;
\item la somme des 3 forces est nulle.
\end{itemize}

\end{resultat}

\subparagraph{}
\textit{On isole l'échelle 3. Réaliser le BAME.}

\subparagraph{}
\textit{Comment se traduit l'application du PFS ?}

\subparagraph{}
\textit{Déterminer les actions mécaniques  en $A$, $B$ et $C$.}

\begin{center}
\includegraphics[width=.7\textwidth]{images/echelle}
\end{center}

\newpage

\subsection*{Ensemble matériel soumis à 3 actions mécaniques parallèles}
\begin{resultat}
Lorsque les actions mécaniques sont parallèles, il est nécessaire d'écrire une équation de moment pour résoudre le système. 
\end{resultat}

\begin{center}
\includegraphics[width=.95\textwidth]{images/remorque}
\end{center}


\subparagraph{}
\textit{Déterminer les actions mécaniques  en $A$, $B$, $C$ et $D$.}

\subsection*{Problème avec frottement}

La voiture proposée et en équilibre dans le position indiquée, les roues avant sont décollées du sol (pas de contact en $A$) et sont en contact en $B$ avec un trottoir de hauteur $h$. Les frottements $B$ et $D$ sont caractérisés par $f_B = f_D = 0,8$. Le poids de la voiture est de $1\,800\; daN$. 

\begin{center}
\includegraphics[width=.8\textwidth]{images/voiture}
\end{center}

 \subparagraph{}
\textit{On considère que les roues arrières seules sont motrices. La voiture peut-elle franchir le trottoir ?}


 \subparagraph{}
\textit{On considère que les roues avant seules sont motrices. La voiture peut-elle franchir le trottoir ?}


 \subparagraph{}
\textit{On considère que les quatre roues sont motrices. Dans quelles conditions a voiture peut-elle franchir le trottoir ?}

\begin{center}
\includegraphics[width=.75\textwidth]{images/voiture}
\end{center}

\begin{center}
\includegraphics[width=.75\textwidth]{images/voiture}
\end{center}

\begin{center}
\includegraphics[width=.75\textwidth]{images/voiture}
\end{center}
\end{document}


