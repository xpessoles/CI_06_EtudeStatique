\documentclass[10pt]{article}
\input{style/coursHeadings}
\input{style/programHeadings}
\input{style/macros_SII}
\input{style/macros_Titres}
\input{style/macros_Frames}

%Si le boolen xp est vrai : compilation pour xabi
%Sinon compilation Damien
\newboolean{xp}
\setboolean{xp}{true}

\newboolean{prof}
\setboolean{prof}{true}

\usepackage[%
    pdftitle={CI 06 : Stat - Problèmes plans},
    pdfauthor={Xavier Pessoles},
    colorlinks=true,
    linkcolor=blue,
    citecolor=magenta]{hyperref}


\def\discipline{Sciences Industrielles de l'Ingénieur}
\def\xxtitre{\ifthenelse{\boolean{xp}}{
CI 6 : Étude du comportement statique des systèmes}{}}

\def\xxsoustitre{\ifthenelse{\boolean{xp}}{
Chapitre 2 --  -- TD}{
Partie  -- }}

\def\xxauteur{\ifthenelse{\boolean{xp}}{
Xavier \textsc{Pessoles}
}{}}

\def\xxpied{\ifthenelse{\boolean{xp}}{
CI 6 : Statique\\
Ch. 3 : Statique plane -- TD 3 -- Arc Boutement}{
\xxtitre}}

\def\xxcathegorie{\ifthenelse{\boolean{xp}}{
2013 -- 2014 \\
Xavier \textsc{Pessoles}}{}}

%---------------------------------------------------------------------------

\begin{document}

\ifthenelse{\boolean{xp}}{\input{style/enteteXP}}{\input{style/enteteDI}}

\begin{center}
\Large{\textsc{Travaux Dirigés -- Résolution graphique des problèmes de statique}}
\end{center}



\section*{Serrage à leviers}
\subsection*{Présentation}

Un montage d’usinage est un dispositif qui assure 2 fonctions principales : 
\begin{itemize}
\item mise en position de la pièce qui va être usinée. 
\item maintien en position de celle-ci lors de l’usinage. 
\end{itemize}
 
Pour notre système : 
\begin{itemize}
\item la mise en position est réalisée à l’aide de plusieurs contacts ponctuels entre le bâti 5 et la pièce 6. 
\item le maintien en position est réalisé par l’intermédiaire d’un système de serrage constitué de 3 leviers 1, 
2 et 3, et d’un vérin 4-7. 
\end{itemize}
 
 
\subsection*{Hypothèses}
\begin{itemize}
\item Le bâti 5 est seulement en liaison avec la pièce 6, le levier 1 en B et le levier 3 en F. 
\item Toutes les liaisons sont supposées parfaites et le poids des pièces est négligé devant les autres 
actions. 
\item L’action de serrage souhaitée du levier 1 sur la pièce 6 est de 1000 N. 
\end{itemize}

\subsection*{Travail demandé}

\subparagraph{}
\textit{Peut-on faire une résolution graphique ?}

\subparagraph{}
\textit{Déterminer le graphe de structure (approprié à une résolution graphique). }

\subparagraph{}
\textit{Donner le cheminement pour déterminer graphiquement l’action que doit exercer la tige du 
vérin 7 sur le levier 3 pour assurer le serrage souhaité. }

\subparagraph{}
\textit{Appliquer cette démarche et déterminer graphiquement, dans la position donnée, cette action.}

\subparagraph{}
\textit{Sachant que le diamètre extérieur du piston vaut 35 mm et que le diamètre de la tige du 
piston vaut 10 mm, en déduire la pression nécessaire dans la chambre du piston. }



\begin{center}
\includegraphics[width=.8\textwidth]{images/fig_01}
\end{center}


\newpage
\section*{Pince à mors parallèles}
\setcounter{subparagraph}{0}
\subsection*{Présentation}

A l’aide d’une structure de forme parallélogramme (ECDF), les deux mors (doigts 5) de la pince ci-dessous 
restent toujours parallèles pendant le serrage de la pièce. 
L’effort développé par le vérin est transmis aux doigts 5 par la biellette 2 et la bielle 6. 

\begin{center}
\includegraphics[width=.8\textwidth]{images/fig_02}
\end{center}

\subsection*{Hypothèses}

\begin{itemize}
\item Le mécanisme est supposé plan. 
\item L’action mécanique en I de la pièce sur le doigt 5 est modélisable par un glisseur vertical de norme 
100 N, l’action sur l’autre doigt est identique. 
\item La pince est symétrique. 
\end{itemize}


\subsection*{Travail demandé}

\subparagraph{}
\textit{Peut-on faire une résolution graphique ? }

\subparagraph{}
\textit{Déterminer le graphe de structure (approprié à une résolution graphique).}

\subparagraph{}
\textit{Donner le cheminement pour déterminer graphiquement l’action du fluide sur la tige. }

\subparagraph{}
\textit{Appliquer cette démarche et déterminer graphiquement, dans la position donnée, cette action. 
(Justifier les différentes étapes de la construction).}

\subparagraph{}
\textit{La barre 4 est-elle soumise à de la traction ou de la compression ?}


\newpage
\section*{Système de pesée embarquée}
\setcounter{subparagraph}{0}
\subsection*{Présentation}
On s’intéresse à un système de collecte et de pesée embarquée 
des ordures ménagères. 

\begin{center}
\includegraphics[width=.8\textwidth]{images/fig_03}
\end{center}

Afin de réduire la quantité des ordures, de plus en plus de municipalités décident de faire payer aux usagers 
une taxe sur les ordures ménagères proportionnelle au poids des déchets. Les camions poubelles sont alors 
équipés de systèmes de pesage de manière à enregistrer le poids des déchets lors du vidage des 
conteneurs. 

Chaque conteneur à ordures comporte en outre un dispositif d’identification unique permettant d’établir 
automatiquement les factures. L’identité du conteneur est donnée à l’aide d’un code à barre. 
 
Un calculateur soustrait après mesures le poids du conteneur vidé au poids du conteneur plein et stocke le 
poids net avec l’identification du conteneur. Les informations récupérées sont alors transmises au système 
informatique pour facturation et analyse. 
 
 
Le système mécanique est donné sur le document page suivante. 
 
 
Le fonctionnement est le suivant : 
 
\subsubsection*{Levage du conteneur}
Le conteneur 8 est soulevé par le peigne 7 qui est lui-même articulé en H sur la bielle 5 et en J sur la bielle 
6. L’effort de levage est fourni par le vérin [3+4]. Celui-ci est articulé en B sur le bras 2 et en F sur la bielle 6. 
C’est lors de cette phase que la mesure de poids est effectuée. Avant utilisation, le système est étalonné en 
utilisant des poids connus. La mesure faite dans le bras de soulèvement 5 est enregistrée sous un angle 
prédéterminé, par exemple dans la position de la figure page suivante, à la fois lors de la course en montée 
et en descente. Pendant cette phase le vérin [0+1] ne fonctionne pas. 
 
 
 
\subsubsection*{Vidage du conteneur}
Le système est animé par le vérin hydraulique [0+1] grâce au système de transformation de mouvement 
pignon-crémaillère. La pièce 2 est en liaison pivot d’axe 
$(A,\vect{z})$ avec la benne 0, ce qui permet le 
basculement du conteneur 8. Lors de cette phase le conteneur est vidé. 
 
 
\subsubsection*{Repose du conteneur}
On fait tourner le conteneur en sens inverse jusqu’à ce qu’il soit dans sa position droite et on l’abaisse 
jusqu’au sol. 


\subsection*{Levage du conteneur}
On rappelle que lors de la phase de soulèvement, seul le vérin [3+4] est actionné. 

\begin{obj}
Dans cette partie, on se propose de déterminer la poussée du vérin [3+4], $\vectf{4}{6}$ , lors de la phase de levée. 
\end{obj}

\begin{hypo}
\begin{itemize}
\item Les liaisons sont considérées comme parfaites. 
\item L’action de la pesanteur sur les différents solides sera négligée sauf sur le conteneur 8 de masse 
100 kg et de centre de gravité G . 
\item Le problème est considéré comme plan $\left(O,\vect{x_0},\vect{y_0} \right)$.
\end{itemize}
\end{hypo}

\textbf{Données :}
\begin{itemize}
\item Le système est en équilibre par rapport à un référentiel galiléen dans la position de la figure dans le 
plan de symétrie $\left(O,\vect{x_0},\vect{y_0} \right)$. 
\item Les pièces 4 et 5 ne se touchent pas. 
\end{itemize}

\textbf{Directive}
Échelle des actions conseillées : 5cm pour 1000 daN.

\subparagraph{}
\textit{Déterminer graphiquement, dans la position donnée, la poussée $\vectf{4}{6}$ du vérin [3+4]. 
(Justifier les différentes étapes de la construction). 
}

\subsection*{Vidage du conteneur}

Lors de cette phase, seul le vérin [0+1] est actionné. L’énergie nécessaire au fonctionnement du système est 
d’origine hydraulique. Elle est fournie par une pompe de débit $300\; cm^3 \cdot .s^{-1}$ entraînée en rotation par le moteur 
du camion. 

\begin{obj}
Dans cette partie, on se propose de vérifier les conditions de vidage du conteneur.
\end{obj}

\textbf{Données}
\begin{itemize}
\item Le corps du vérin 0 est immobile par rapport au camion-benne. 
\item Le mouvement de la tige du vérin 1 par rapport au camion-benne 0 est une translation de direction 
horizontale. 
\item La tige du vérin [0+1] a une course de 50 mm, le rayon du piston est de 65 mm. 
\item Le rayon primitif du pignon 2 est de 20 mm. 
\end{itemize}

\subparagraph{}
\textit{Compte tenu de la course de la tige du vérin 1, déterminer l’angle de basculement du conteneur. }

\begin{center}
\includegraphics[width=.3\textwidth]{images/fig_04}
\end{center}

\subparagraph{}
\textit{Dans cette position, en déduire l’angle de frottement maximal en deçà duquel les déchets 
pourraient rester à l’intérieur du conteneur. Conclure quant au vidage du conteneur. }

\subparagraph{}
\textit{Déterminer la vitesse de déplacement du vérin ainsi que le temps nécessaire au basculement.}

\begin{center}
\includegraphics[width=.8\textwidth]{images/fig_05}
\end{center}


\newpage
\section*{Skip de chargement}
\setcounter{subparagraph}{0}
\subsection*{Présentation}

Le skip de chargement étudié est utilisé dans les carrières pour l’extraction 
des matières premières. 
 
Il se compose d’un chariot (skip) 3 guidé par deux rails (profilés en U) 
solidaires d’un châssis fixe 1. 
Le guidage du chariot sur les rails est réalisé par deux paires de galets 4 et 5 
en liaison pivot avec le chariot, et en contact en A et B avec les rails. 
 
Le levage est réalisé à l’aide d’une chaîne entraînée par un motoréducteur 
(non représenté). La capacité de levage est de 5000 N. 
 
On se propose de rechercher les actions en A et B ainsi que la tension dans la 
chaîne en vue de son dimensionnement. 

\begin{center}
\includegraphics[width=.3\textwidth]{images/fig_06}
\end{center}

\begin{hypo}
\begin{itemize}
\item Les liaisons sont considérées comme parfaites. 
\item L’action de la pesanteur sur les différents solides sera négligée sauf sur le chariot 3 de masse 
M = 500 kg et de centre de gravité $G$. On prendra $g=10 m\cdot s^{-2}$.
\item Le problème est considéré comme plan $\left( O,\vect{x},\vect{y}\right)$.
\item Le système est en équilibre dans la position de la figure de la page suivante. 
\item Le frottement ainsi que la résistance au roulement en A et B sont négligés. 
\item On suppose avoir isolé la chaîne de façon à connaître la droite d’action en C. 
\item On suppose avoir isolé les galets de façon à connaître la droite d’action en A et B. 
\end{itemize}
\end{hypo}

\subparagraph{}
\textit{Déterminer les actions mécaniques en A, B et C appliquées sur le skip. }

\begin{center}
\includegraphics[width=.6\textwidth]{images/fig_07}
\end{center}

\end{document}


\newpage
\section*{}
\setcounter{subparagraph}{0}
\subsection*{}



\begin{itemize}
\end{itemize}
\end{document}


