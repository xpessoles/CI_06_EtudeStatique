\documentclass[10pt]{article}
\input{style/coursHeadings}
\input{style/programHeadings}
\input{style/macros_SII}
\input{style/macros_Titres}
\input{style/macros_Frames}

%Si le boolen xp est vrai : compilation pour xabi
%Sinon compilation Damien
\newboolean{xp}
\setboolean{xp}{true}

\newboolean{prof}
\setboolean{prof}{false}

\usepackage[%
    pdftitle={CI 06 : Stat - Modélisation des AM},
    pdfauthor={Xavier Pessoles},
    colorlinks=true,
    linkcolor=blue,
    citecolor=magenta]{hyperref}


\def\discipline{Sciences Industrielles de l'Ingénieur}
\def\xxtitre{\ifthenelse{\boolean{xp}}{
CI 06 : Étude du comportement statique des systèmes}{}}

\def\xxsoustitre{\ifthenelse{\boolean{xp}}{
Chapitre 1 -- Modélisation des Actions Mécaniques}{
Partie  -- }}

\def\xxauteur{\ifthenelse{\boolean{xp}}{
%Xavier \textsc{Pessoles}
}{}}

\def\xxpied{\ifthenelse{\boolean{xp}}{
CI 06 : Statique\\
Ch. 2 : PFS -- TD -- Borne rétractable}{
\xxtitre}}

\def\xxcathegorie{\ifthenelse{\boolean{xp}}{
2013 -- 2014 \\
Xavier \textsc{Pessoles}}{}}





%---------------------------------------------------------------------------


\begin{document}

\ifthenelse{\boolean{xp}}{\input{style/enteteXP}}{\input{style/enteteDI}}

\begin{center}
\Large{\textsc{Travaux Dirigés : Principe fondamental de la statique}}
\end{center}
\begin{flushright}
\textit{D'après concours ATS 2010.}
\end{flushright}
\vspace{.5cm}

\section*{Borne rétractable}
\begin{minipage}[c]{.4\linewidth}
\begin{center}
\includegraphics[width=.8\textwidth]{images/borne}
\end{center}
\end{minipage} \hfill
\begin{minipage}[c]{.59\linewidth}
Le dispositif étudié est un système permettant de limiter ou d’interdire la circulation
dans des zones à accès réservé. Ce dispositif comporte :
\begin{itemize}
\item un caisson intégrant la partie opérative, à savoir une borne motorisée rétractable dans le sol;
\item un caisson intégrant la partie commande comportant :
\begin{itemize}
\item une platine électronique de gestion;
\item une batterie d’alimentation électrique du système;
\item des cellules photovoltaïques assurant la charge de la batterie.
\end{itemize}
\end{itemize}
\end{minipage} 

\begin{minipage}[c]{.3\linewidth}
On donne ci-contre le schéma d'architecture du mécanisme.

\vspace{.5cm}

\begin{itemize}
\item 0 : bâti
\item 1 : chariot
\item 2 : motoréducteur
\item 3 : pignon
\end{itemize}

\vspace{.5cm}

\begin{itemize}
\item $\vect{OA} = - \dfrac{h}{2} \vect{y}$
\item $\vect{OB} = \dfrac{h}{2} \vect{y}$
\item $\vect{OC} = l \vect{z}$
\item $\vect{OG} = d \vect{x} + L\vect{z}$
\end{itemize}

\vspace{.5cm}
La masse de la borne (enemble chariot) est de 80 kg.

\end{minipage} \hfill
\begin{minipage}[c]{.6\linewidth}
\begin{center}
\includegraphics[width=.9\textwidth]{images/schema}
\end{center}
\end{minipage} 

Afin de limiter les efforts résistants liés aux frottements dans les guidages en translation du chariot, le constructeur a choisi de placer un contrepoids qui permet de positionner le centre de gravité $G$ de la partie mobile liée au chariot à la distance
$d$ de la ligne de référence de la crémaillère.

On se propose d’étudier la position du contrepoids permettant de minimiser les pertes par frottement dans le guidage du chariot 1 et ainsi
augmenter l’autonomie du système.
\begin{hypo}
\begin{itemize}
\item La détermination de la position du contrepoids est effectuée pour la montée à
vitesse constante ce qui justifie une étude en statique.
%\item Les frottements ne sont pas négligés dans les liaisons pivot glissant
%constituant le guidage du chariot par rapport au bâti. On prendra un facteur de frottement $\tan \varphi = 0,22$.
\item Le poids du chariot et de tous les éléments embarqués (motoréducteur, borne,
etc.) n’est pas négligé. On considère la masse totale : $m = 30\;  kg$ et
l’accélération de la pesanteur : $g = 10 \; m\cdot s^{-2}$.
\item On suppose que les résultantes des actions mécaniques transmissibles par
les liaisons en $A$ et $B$ sont situées respectivement dans les plans $(A,\vect{x},\vect{z})$ et
$(B,\vect{x},\vect{z})$ . En outre, elles présentent une symétrie par rapport au plan $(O,\vect{x},\vect{z})$.
\end{itemize}
\end{hypo}
On donne le torseur des actions mécaniques exercées par 0 sur 3 au point $C$ :

$$
\torseurstat{T}{0}{3}=\torseurl{X_{03}\vect{x}+Z_{03}\vect{z}}{\vect{0}}{C,\mathcal{R}} \quad \text{avec } \tan\alpha = -\dfrac{X_{03}}{Z_{03}}
$$


\subparagraph{}
\textit{Quelle disposition constructive (quelle solution technologique) proposeriez-vous pour satisfaire l'exigence ***. Comment l'integreriez-vous dans le dispositif proposé ?}

\subparagraph{}
\textit{Réalier le graphe des liaisons du système. Indiquez sur ce graphe les actions mécaniques extérieures. }

\subparagraph{}
\textit{On isole l'ensemble $E=\{1+2+3\}$. Réaliser le bilan des actions mécaniques s'exerçant sur $E$. }


\subparagraph{}
\textit{Faire l'inventaire des grandeurs inconnues. Combien d'inconnues peut permettre le PFS ? Que pouvez-vous en conclure ?}

\subparagraph{}
\textit{Appliquer le PFS en $O$.}


\subparagraph{}
\textit{Déterminer les inconnues. }


\end{document}


