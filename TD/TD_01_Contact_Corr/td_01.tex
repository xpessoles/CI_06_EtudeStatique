\documentclass[11pt,oneside]{article}
\input{coursHeadings}

\usepackage[%
    pdftitle={TD Statique},
    pdfauthor={Xavier Pessoles},
    colorlinks=true,
    linkcolor=blue,
    citecolor=magenta]{hyperref}



% \makeatletter \let\ps@plain\ps@empty \makeatother
%% DEBUT DU DOCUMENT
%% =================
\sloppy
\hyphenpenalty 10000

\newcommand{\Pointilles}[1][3]{%
\multido{}{#1}{\makebox[\linewidth]{\dotfill}\\[\parskip]
}}


\begin{document}


\newboolean{prof}
\setboolean{prof}{false}
%------------- En tetes et Pieds de Pages ------------
\pagestyle{fancy}
\renewcommand{\headrulewidth}{0pt}

\fancyhead{}
\fancyhead[L]{%
\begin{minipage}[c]{1.6cm}
\includegraphics[width=1.4cm]{png/logo_jh_ptsi.png}%
\end{minipage}
\rule{2cm}{.5pt}
}

\fancyhead[C]{\rule{12cm}{.5pt}}

\fancyhead[R]{%
\begin{minipage}[c]{3cm}
\begin{flushright}
\footnotesize{\textit{\textsf{Sciences Industrielles\\ pour l'Ingénieur}}}%
\end{flushright}
\end{minipage}
}

\renewcommand{\footrulewidth}{0.2pt}

\fancyfoot[C]{\footnotesize{\bfseries \thepage}}
\fancyfoot[L]{\footnotesize{2011 -- 2012} \\ X. \textsc{Pessoles}}
\ifthenelse{\boolean{prof}}{%
\fancyfoot[R]{\footnotesize{TD -- CI 3 : Statique -- P}}
}{%
\fancyfoot[R]{\footnotesize{TD -- CI 3 : Statique}}
}


%\begin{center}
%\textit{Centre d'intérêt}
%\end{center}

\begin{center}
 \huge\textsc{CI 3 -- Statique : Modélisation, prévision et vérification du comportement statique des systèmes}
\end{center}

\begin{center}
 \LARGE\textsc{Chapitre 1 -- Modélisation des actions mécaniques} 
\end{center}

\vspace{.5cm}

\section*{Modélisation des actions mécaniques de contact sur un palier lisse}
\begin{minipage}[c]{.55\linewidth}

On souhaite déterminer le modèle global des actions mécaniques de contact sur un palier lisse, composant technologique pour le guidage en rotation.

On donne le modèle local :
\begin{itemize}
\item les surfaces de contact sont limitées par un demi cylindre de longueur $L$ et de rayon $R$;
\end{itemize}

\textbf{On considère dans un premier temps que la répartition de pression est uniforme.}

La pression $p$ est uniforme sur chaque élément $dS$ situé autour du point $M$.

\end{minipage}\hfill
\begin{minipage}[c]{.4\linewidth}
\begin{center}
\includegraphics[width=.9\textwidth]{png/fig5}
\end{center}
\end{minipage}

\vspace{.25cm}
\begin{center}
\includegraphics[width=.75\textwidth]{png/fig6}
\end{center}

\setcounter{paragraph}{0}
\paragraph{}
\textit{Déterminer le modèle global de l'action mécanique de l'arbre 2 sur le bâti 1 sous forme d'un torseur exprimé au point $O$.}

Exprimons le torseur des actions mécaniques sous sa forme locale en un point $M$ : 

$$
\torseurl{d\vectf{2}{1}}{d\vectm{M}{2}{1}=\vect{0}}{M}
$$

La forme globale au point O est alors donnée par :

$$
\torseurstat{T}{2}{1} = \torseurl{\vectf{2}{1} = \int d\vectf{2}{1}}{\vectm{M}{2}{1} = \int d\vectm{M}{2}{1}= \int \vect{OM}\wedge d\vectf{2}{1}}{M}
$$

\vspace{.5cm}

\textbf{Calculons $\vectf{2}{1}$.}

$$
\vectf{2}{1} = \int d\vectf{2}{1} = \iint p \vect{-r} dS = -p \iint  \vect{r} dS
= -p \iint  \left(\cos\theta\vect{x}+\sin\theta\vect{y} \right) dS $$

$$
\vectf{2}{1}
= -p \int\limits_{-L/2}^{L/2} \int\limits_{-\pi/2}^{\pi/2}   \left(\cos\theta\vect{x}+\sin\theta\vect{y} \right) Rd\theta dz
= -p R L \int\limits_{-\pi/2}^{\pi/2}   \left(\cos\theta\vect{x}+\sin\theta\vect{y} \right) d\theta 
$$
$$
\vectf{2}{1}
= -p R L \left(\int\limits_{-\pi/2}^{\pi/2}   \cos\theta\vect{x}d\theta + \int\limits_{-\pi/2}^{\pi/2} \sin\theta\vect{y} d\theta \right)
= -p R L \left(\left[\sin\theta \right]_{-\pi/2}^{\pi/2}\vect{x}
+\left[ -\cos\theta\right]_{-\pi/2}^{\pi/2}\vect{y}
\right)
$$

$$
\vectf{2}{1}
= -p R L \left(2\vect{x}
+0\vect{y}
\right) = -2pRL\vect{x}
$$

$2RL$ est appelée surface projetée du cylindre. Elle correspond au produit du diamètre par sa longueur.

\vspace{.5cm}

\textbf{Calculons $\vectm{M}{2}{1}$.}
$$
\vectm{M}{2}{1} = \int d\vectm{M}{2}{1}= \int \vect{OM}\wedge d\vectf{2}{1}
$$

$$
\vectm{M}{2}{1} = -p \iint R\vect{r} \wedge \vect{r}dS = \vect{0}
$$

Au final, 
$$
\torseurstat{T}{2}{1} = \torseurl{\vectf{2}{1} = -2pRL\vect{x}}{\vectm{M}{2}{1} =  \vect{0}}{M}
$$


\paragraph{}
\textit{Calculer $\vectf{2}{1}$ lorsque la pression est de la forme : $p(\theta)=p_0\cos\theta$ pour $\theta\in[-\pi/2,\pi/2]$.}

Dans ce cas : 
$$
\vectf{2}{1} = \int d\vectf{2}{1} = \iint p(\theta) \vect{-r} dS 
= - p_0R\iint\cos\theta  \left(\cos\theta\vect{x}+\sin\theta\vect{y} \right)  d\theta dz$$

$$
\vectf{2}{1} 
= - p_0 L R\int\limits_{-\pi/2}^{\pi/2}\cos\theta  \left(\cos\theta\vect{x}+\sin\theta\vect{y} \right)  d\theta$$

$$
\int\limits_{-\pi/2}^{\pi/2}\cos^2\theta  d\theta = \dfrac{\pi}{2}
\quad 
\text{et}
\quad
\int\limits_{-\pi/2}^{\pi/2}\cos\theta \sin\theta  d\theta = 0
$$
Au final :
$$
\vectf{2}{1} 
= - p_0 L R \dfrac{\pi}{2}\vect{x}$$



\section*{Couple transmis par une clavette}

\begin{minipage}[c]{.65\linewidth}

On cherche à connaître le couple transmissible autour de $\vect{z}$, axe du pignon.

La clavette est de hauteur $h$ et de largeur $l$. On note $p$ le champ de pression uniforme du pignon sur la clavette. $p$ est appelée pression de matage. 

$O$ est un point de l'axe.
\end{minipage} \hfill
\begin{minipage}[c]{.3\linewidth}
\begin{center}
\includegraphics[width=\textwidth]{png/clavette}
\end{center}
\end{minipage} 


$$
\vectm{O}{\text{Pignon}}{\text{Arbre}} \cdot \vect{z} = 
\vectm{O}{P}{A} \cdot \vect{z} 
= \int \vect{OM}\wedge d\vectf{P}{A} \cdot \vect{z}
= \int x\vect{x} \wedge p \vect{y}  dx dz \cdot \vect{z}
$$

$$
\vectm{O}{\text{Pignon}}{\text{Arbre}} \cdot \vect{z} 
= \int\limits_{0}^{l}\int\limits_{R-h/2}^{R+h/2} px\underbrace{\left( \vect{x} \wedge  \vect{y}\right)\cdot \vect{z}}_1  dx dz 
= pl\int\limits_{R-h/2}^{R+h/2} x  dx =\dfrac{1}{2}pl\left(\left(R+h/2 \right)^2-\left( R-h/2\right)^2 \right)
$$


$$
\vectm{O}{\text{Pignon}}{\text{Arbre}} \cdot \vect{z} 
=hplR
$$

Ce résultat peut paraître logique : la force exercée sur la clavette s'exprime par $phl$. Le bras de levier du glisseur correspond au rayon de l'arbre auquel on ajoute un quart de hauteur de clavette.


Attention, si on considère que le couple est transmis par l'ensemble du flanc de la clavette, le moment transmissible est de la forme : 

$$
\vectm{O}{\text{Pignon}}{\text{Arbre}} \cdot \vect{z} 
=hLpR
$$
\end{document}