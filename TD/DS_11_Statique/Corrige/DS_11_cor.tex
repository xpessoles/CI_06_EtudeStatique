\documentclass[10pt,oneside]{article}
\input{coursHeadings}

\usepackage[%
    pdftitle={Statique - DS11},
    pdfauthor={Xavier Pessoles},
    colorlinks=true,
    linkcolor=blue,
    citecolor=magenta]{hyperref}



% \makeatletter \let\ps@plain\ps@empty \makeatother
%% DEBUT DU DOCUMENT
%% =================
\sloppy
\hyphenpenalty 10000

\newcommand{\Pointilles}[1][3]{%
\multido{}{#1}{\makebox[\linewidth]{\dotfill}\\[\parskip]
}}

\begin{document}


\newboolean{prof}
\setboolean{prof}{false}
%------------- En tetes et Pieds de Pages ------------
\pagestyle{fancy}
\renewcommand{\headrulewidth}{0.2pt}

\fancyhead{}
\fancyhead[L]{PTSI -- Sciences Industrielles pour l'Ingénieur}
\fancyhead[R]{Lycée Jules Haag -- Besançon}


\renewcommand{\footrulewidth}{0.2pt}
\fancyfoot[C]{\bfseries \thepage}
\fancyfoot[L]{2010 -- 2011}
\ifthenelse{\boolean{prof}}{%
\fancyfoot[R]{DS 11 -- Statique}
}{%
\fancyfoot[R]{DS 11 -- Statique}
}

%\fancyfoot[RO]{Version du \today}
% \fancyfoot[LE]{Version du \today}
% \fancyfoot[RO]{\textcolor{gris25}{Version rapporteurs}}
% \fancyfoot[LE]{\textcolor{gris25}{Version rapporteurs}}
% \fancyfoot[R]{\textcolor{gris25}{Version rapporteurs}}
% ----------------------------------------------------



\vspace{1cm}

\begin{center}
 \huge\textsc{Devoir Surveillé 11}

\vspace{1cm}

 \large\textsc{Éléments de corrigés}
\end{center}

\vspace{1cm}


\noindent\rule{\linewidth}{.2pt}
\begin{center}
 \large\textbf{CI : 3} \textit{Statique : Modélisation, prévision et vérification du comportement statique des systèmes}

\end{center}
\noindent\rule{\linewidth}{.2pt}



\section{Système de positionnement de radar}
\subsection*{Etude du critère de masse à déplacer}

\paragraph{}
\begin{center}
\includegraphics[width=.6\textwidth]{png/img1}
\end{center}

\paragraph{}
\textit{Donner la forme des torseurs d'actions transmissibles des liaisons 1--3, 3--4 et 2--4 dans la base 3 conformément à la notation imposée ci-dessous.}
La liaison entre 1 et 3 est une liaison rotule de centre $C$: 

$$
\left\{
F_{1 \rightarrow 3} 
\right\}=
\left\{
\begin{array}{cc}
X_{13} & 0 \\
Y_{13} & 0 \\
Z_{13} & 0 \\
\end{array}
\right\}_{C,B_3}
$$

La liaison entre 3 et 4 est une liaison pivot glissant de de centre $F$ et d'axe $\vect{x_3}$: 

$$
\left\{
F_{4 \rightarrow 3} 
\right\}=
\left\{
\begin{array}{cc}
0 & 0 \\
Y_{43} & M_{43} \\
Z_{43} & N_{43} \\
\end{array}
\right\}_{F,B_3}
$$

La liaison entre 2 et 4 est une liaison rotule de de centre $D$ : 

$$
\left\{
F_{2 \rightarrow 4} 
\right\}=
\left\{
\begin{array}{cc}
X_{24} & 0 \\
Y_{24} & 0 \\
Z_{24} & 0 \\
\end{array}
\right\}_{D,B_3}
$$
\paragraph{}
\textit{Déterminer les directions de $\vect{R_{1\rightarrow 3}}$ et de $\vect{R_{2\rightarrow 4}}$ puis en déduire des simplifications dans les torseurs précédents.}

On isole les solides $3$ et $4$. Cet ensemble est soumis à 2 forces. D'après le PFS, ces deux forces ont même norme, même direction et sens opposé. En conséquence, $X_{13}+X_{24}=0$ et $Y_{13}=Z_{13}=Y_{24}=Z_{24}=0$.

$$
\left\{
F_{1 \rightarrow 3} 
\right\}=
\left\{
\begin{array}{cc}
X_{13} & 0 \\
0 & 0 \\
0 & 0 \\
\end{array}
\right\}_{C,B_3}
\quad 
\left\{
F_{2 \rightarrow 4} 
\right\}=
\left\{
\begin{array}{cc}
X_{24} & 0 \\
0 & 0 \\
0 & 0 \\
\end{array}
\right\}_{D,B_3}
$$


\paragraph{}
\textit{Déterminer l'expression de $||\vect{R_{2\rightarrow 4}}||$ en fonction de $p$ et $S$.}
On a $||\vect{R_{2\rightarrow 4}}||=pS$ 

\paragraph{}
\textit{En isolant le solide $2$ et en utilisant le théorème du moment statique au point $A$ projetée sur $\vect{z_2}$, déterminer l'expression de $X_{42}$ en fonction de $P$ et des paramètres géométriques utiles.}
On isole le solide 2. 

On fait le bilan des actions mécaniques extérieures au point $A$ :
Liaison rotule en $A$ : 
$$
\left\{
F_{1 \rightarrow 2} 
\right\}=
\left\{
\begin{array}{cc}
X_{12} & 0 \\
Y_{12} & 0 \\
Z_{12} & 0 \\
\end{array}
\right\}_{A,B_2}
$$



Liaison linéaire annulaire au point $B$ :
$$
\left\{
F_{1 \rightarrow 2} 
\right\}=
\left\{
\begin{array}{cc}
0 & 0 \\
Y'_{12} & 0 \\
Z'_{12} & 0 \\
\end{array}
\right\}_{B,B_2}
=
\left\{
\begin{array}{cc}
0 &  -\\
Y'_{12} &  -\\
Z'_{12} &  0\\
\end{array}
\right\}_{A,B_2}
$$


Liaison rotule au point $D$ :
$$
\left\{
F_{4 \rightarrow 2} 
\right\}
=
\left\{
\begin{array}{cc}
X_{42}  & 0 \\
0 & 0 \\
0 & 0 \\
\end{array}
\right\}_{D,B_2}
=
\left\{
\begin{array}{cc}
X_{42}  & -\\
0 & -\\
0 & 0 \\
\end{array}
\right\}_{A,B_2}
$$


Pesanteur en $G$ :
$$
\left\{
F_{pes \rightarrow 2} 
\right\}
=
\left\{
\begin{array}{cc}
-P  & 0 \\
0 & 0 \\
0 & 0 \\
\end{array}
\right\}_{G,B_2}
=
\left\{
\begin{array}{cc}
-P  & -\\
0 & -  \\
0 &  \\
\end{array}
\right\}_{A,B_2}
$$

\paragraph{}
\textit{Calculer la valeur de la pression $p$ maximale pour la position $\theta=0^o$ (et donc $\gamma = 18^o$) et $\beta = 0^o$ avec les valeurs numériques données.}


\paragraph{}
\textit{Le vérin utilisé peut supporter une pression de 10 bars maximum. Conclure quant à la capacité du système à satisfaire le critère de masse de radar à déplacer.}



\section*{Griffe et lame de bulldozer}
\setcounter{paragraph}{0}


La lame 2 est rattachée au bulldozer 1 par l'intermédiaire de la pièce 3 ainsi que les deux vérins 7+6 et 5+4. La griffe 13 est rattachée au bulldozer par l'intermédiaire de la pièce 12 et du vérin 8+9. Les liaisons aux points $A$, $B$, $C$, $D$, $E$, $F$, $G$, $H$, $I$ et $J$ sont des liaisons pivots parfaites suivant l'axe $\vect{z_0}$. La pièce 12 est reliée à la griffe 13 au point $K$ grâce à une rainure. 

Tous les vérins ont une surface de piston identique de $2\, 500\pi\; mm^2$.

\paragraph{}
\textit{La terre exerce sur la griffe une action mécanique $\vect{F_{\text{sol}\rightarrow \text{griffe}}}$ au point $M$ donnée sur le document réponse. Résoudre graphiquement le problème pour déterminer la pression dans les deux vérins actionnant sur la griffe.}

\textbf{Pour les deux premières questions, vous énoncerez brièvement la démarche utilisée. De plus, vous indiquerez clairement sur le dessin les directions des efforts que vous tracez.}

\paragraph{}
\textit{La terre exerce sur la lame une action mécanique $\vect{F_{\text{sol}\rightarrow \text{lame}}}$ au point $N$ donnée sur le document réponse. Résoudre graphiquement le problème pour déterminer la pression dans les deux vérins actionnant la lame.}


\paragraph{}
\textit{Conclure vis-à-vis du cahier des charges quant aux performances obtenues.}



\section*{Echelle en appui sur un mur}

\setcounter{paragraph}{0}
\paragraph{Tous les frottements sont négligés}
\textit{Le grimpeur est situé au milieu de l'échelle. En appliquant le PFS en A à l'ensemble échelle et grimpeur, sans écrire le moindre torseur, montrer que l'échelle ne peut tenir en équilibre.}

\paragraph{On considère qu'il y a du frottement au point $A$.}
\textit{Le coefficient de frottement est égal à 0,3. Reproduire le schéma et faire apparaître le cône de frottement. Préciser la position de l'effort normal et de l'effort tangentiel à la limite du glissement.}
 

\paragraph{On considère qu'il y a du frottement au point $A$.}
\textit{Appliquer le PFS au point A à l'ensemble échelle et grimpeur, déterminer l'angle $\alpha$ pour lequel l'échelle est en équilibre.}

\paragraph{On considère qu'il y a du frottement au point $A$ et en $B$.}
\textit{Dans quelle situation se trouve le système. Pour $\alpha=20^o$, préciser pour quel poids P le système est en équilibre. Vous pourrez utiliser une méthode graphique.}

\end{document}







