\documentclass[10pt]{article}
\input{style/coursHeadings}
\input{style/programHeadings}
\input{style/macros_SII}
\input{style/macros_Titres}
\input{style/macros_Frames}

%Si le boolen xp est vrai : compilation pour xabi
%Sinon compilation Damien
\newboolean{xp}
\setboolean{xp}{true}

\newboolean{prof}
\setboolean{prof}{false}

\usepackage[%
    pdftitle={CI 06 : Stat - Modélisation des AM},
    pdfauthor={Xavier Pessoles},
    colorlinks=true,
    linkcolor=blue,
    citecolor=magenta]{hyperref}


\def\discipline{Sciences Industrielles de l'Ingénieur}
\def\xxtitre{\ifthenelse{\boolean{xp}}{
CI 06 : Étude du comportement statique des systèmes}{}}

\def\xxsoustitre{\ifthenelse{\boolean{xp}}{
Chapitre 1 -- Modélisation des Actions Mécaniques}{
Partie  -- }}

\def\xxauteur{\ifthenelse{\boolean{xp}}{
%Xavier \textsc{Pessoles}
}{}}

\def\xxpied{\ifthenelse{\boolean{xp}}{
CI 06 : Statique\\
Ch. 1 : Modélisation des AM -- TD 2 -- Lois de Coulomb}{
\xxtitre}}

\def\xxcathegorie{\ifthenelse{\boolean{xp}}{
2013 -- 2014 \\
Xavier \textsc{Pessoles}}{}}





%---------------------------------------------------------------------------


\begin{document}

\ifthenelse{\boolean{xp}}{\input{style/enteteXP}}{\input{style/enteteDI}}

\begin{center}
\Large{\textsc{Travaux Dirigés : Modélisation des Actions Mécaniques}}
\end{center}
\begin{flushright}
\textit{D'après Ressources de JP Pupier.}
\end{flushright}
\vspace{.5cm}



\subsection*{Renvoi d'angle}
\ifthenelse{\boolean{prof}}{}{}

Ce renvoi d'angle permet de faire tourner un arbre vertical à partir du mouvement d'un arbre horizontal 9. Il comprend un engrenage à pignons coniques 5 et 6.


\begin{center}
\includegraphics[width=.9\textwidth]{images/fig_01}
\end{center}



\subparagraph{}
\textit{Faire le schéma architectural du sous ensemble formé par les pièces 1, 9, 10, 11, 7 et 6. Ceci permettra de modéliser correctement les deux roulements en fonction de leur type mais aussi de leur montage.}

On donne $\vect{AB}=50\vect{x}$, $\vect{AC}=95\vect{x}$, $\vect{AD}=131\vect{x}$, $\vect{DE}=28\vect{y}$ (valeurs en mm). 

On note $\torseurstat{T}{5}{6}$ le torseur d'efforts qu'exerce le pignon 5 sur le pignon 6. Il s'agit d'une force de point d'application E perpendiculaire à la surface de la denture (voir vue suivant F figure suivante).

LA résultante du  $\torseurstat{T}{5}{6}$ est composé des trois forces perpendiculaires $\vect{A}$, $\vect{R}$ et $\vect{T}$. 

\begin{rem}
\begin{itemize}
\item $A$ pour effort axial, c'est-à-dire parallèle à l'axe du pignon;
\item $R$ pour effort radial : c'est une force qui est perpendiculaire à l'axe du pignon et qui donc le coupe en $D$;
\item $T$ pour effort tangentiel car il est tangent au cône primitif. C'est la seule force utile;
\item $\delta$ est l'angle primitif du pignon conique;
\item $\alpha$ est l'angle de pression de la denture. La valeur de cet angle est normalisée.
\end{itemize}
\end{rem}

\subparagraph{}
\textit{Exprimer  $\torseurstat{T}{5}{6}$ en fonction de $T$, $\alpha$ et $\delta$. }

\begin{hypo}
\begin{itemize}
\item Le poids des pièces est négligé.
\item Le frottement est négligé.
\item Le couple moteur fourni à l'arbre 9 vaut 5 m.daN. Son vecteur moment est colinéaire à x est il de sens négatif sur ce même axe. Il s'applique en A.
\item $\alpha = 20\textdegree$; $\delta = 54\textdegree$.
\end{itemize}
\end{hypo}

\subparagraph{}
\textit{Donner le torseur des actions mécanique de la liaison rotule en $B$.}

\subparagraph{}
\textit{Donner le torseur des actions mécanique de la liaison linéaire annulaire en $C$ puis en $D$. }


\subparagraph{}
\textit{Donner le torseur des actions mécanique du pignon en $E$. }

\subparagraph{}
\textit{Donner le torseur d'un couple moteur pur en $C$ puis en $E$.}

\subparagraph{}
\textit{Faire la somme des torseurs.}


\subparagraph{}
\textit{Résoudre le système d'équations.}
\end{document}




