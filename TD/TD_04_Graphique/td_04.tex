\documentclass[11pt,oneside]{article}
\input{coursHeadings}

\usepackage[%
    pdftitle={TD Statique},
    pdfauthor={Xavier Pessoles},
    colorlinks=true,
    linkcolor=blue,
    citecolor=magenta]{hyperref}



% \makeatletter \let\ps@plain\ps@empty \makeatother
%% DEBUT DU DOCUMENT
%% =================
\sloppy
\hyphenpenalty 10000

\newcommand{\Pointilles}[1][3]{%
\multido{}{#1}{\makebox[\linewidth]{\dotfill}\\[\parskip]
}}


\begin{document}


\newboolean{prof}
\setboolean{prof}{false}
%------------- En tetes et Pieds de Pages ------------
\pagestyle{fancy}
\renewcommand{\headrulewidth}{0pt}

\fancyhead{}
\fancyhead[L]{%
\begin{minipage}[c]{1.6cm}
\includegraphics[width=1.4cm]{png/logo_jh_ptsi.png}%
\end{minipage}
\rule{2cm}{.5pt}
}

\fancyhead[C]{\rule{12cm}{.5pt}}

\fancyhead[R]{%
\begin{minipage}[c]{3cm}
\begin{flushright}
\footnotesize{\textit{\textsf{Sciences Industrielles\\ pour l'Ingénieur}}}%
\end{flushright}
\end{minipage}
}

\renewcommand{\footrulewidth}{0.2pt}

\fancyfoot[C]{\footnotesize{\bfseries \thepage}}
\fancyfoot[L]{\footnotesize{2011 -- 2012} \\ X. \textsc{Pessoles}}
\ifthenelse{\boolean{prof}}{%
\fancyfoot[R]{\footnotesize{TD -- CI 3 : Statique -- P}}
}{%
\fancyfoot[R]{\footnotesize{TD -- CI 3 : Statique}}
}


%\begin{center}
%\textit{Centre d'intérêt}
%\end{center}

\begin{center}
 \huge\textsc{CI 3 -- Statique : Modélisation, prévision et vérification du comportement statique des systèmes}
\end{center}

\begin{center}
 \LARGE\textsc{Chapitre 3 -- Étude graphique} 
\end{center}

%\setcounter{paragraph}{0}
\vspace{.5cm}

\section*{Suspension arrière de VTT}

%\begin{minipage}[c]{.35\linewidth}
\begin{center}
\includegraphics[height=4cm]{png/img1}
\hspace{1.5cm}
\includegraphics[height=4cm]{png/img2}
\end{center}
%\end{minipage}

On se propose d’étudier la transmission des efforts dans une suspension arrière de VTT Kona
« Dawg» de « back country » .

Le basculeur 4 est en liaison pivot avec le cadre, d'une part, et il est articulé sur le hauban 3 et sur le
demi combiné 5 (voir schéma cinématique page suivante et photo de la suspension arrière ci
dessous).
Le cadre du VTT est considéré comme fixe, et on applique à la roue arrière un glisseur vertical $(A,\vect{y})$
au point de contact $A$, de module $700\;N$ par l'intermédiaire du support 7, simulant l’action du sol
lorsque le pilote est assis sur le vélo.

L'exercice consiste à déterminer graphiquement l’effort exercé sur l’ensemble (ressort pneumatique
/ amortisseur) (5+6), puis déterminer la pression d’air à établir sous charge dans le ressort
pneumatique (identique à un vérin simple effet dont l’orifice serait bouché).
Le problème est considéré plan, les pivots transmettent donc des glisseurs.

\paragraph{}
\textit{Tracer $\vect{A_{7\rightarrow2}}$ puis déterminer $\vect{A_{4\rightarrow5}}$ , en expliquant soigneusement la méthode.
Le diamètre du piston du « vérin » de suspension (ressort pneumatique) est de 40 mm.}

\paragraph{}
\textit{Déterminer la pression (relative) régnant dans la chambre sous la charge statique
calculée précédemment, en précisant son unité.}


\begin{center}
\includegraphics[width=.9\textwidth]{png/img3}
\end{center}
\end{document}


\begin{minipage}[c]{.6\linewidth}
Les grues portuaires permettent de transporter des marchandises pour les débarquer des bateaux sur les quais ou pour charger les marchandises dans les bateaux. Ces systèmes sont toujours équipés d'un frein de sécurité qui permet de freiner la chute des objets à porter au cas où un dysfonctionnement apparaîtrait. L'objectif est de vérifier si le frein de sécurité, dont on donne un extrait de cahier des charges ci dessous, permet de satisfaire le niveau du critère de la fonction FS1. 
\end{minipage}\hfill
\begin{minipage}[c]{.35\linewidth}
\begin{center}
\includegraphics[width=.95\textwidth]{png/grue}
\end{center}
\end{minipage}
\begin{center}
\includegraphics[width=.8\textwidth]{png/fonctions}
\end{center}

Le schéma cinématique du frein est fourni sur la figure de la page suivante. L'objet à porter repéré 8 sur le schéma est soumis à la gravité. On néglige la passe de toutes les autres pièces. La pige 9 relie les pièces 2, 3 et 4 au point $B$, toutes en liaison pivot par rapport à la pige 9. 

Toutes les liaisons sont parfaites sauf le contact entre 5 et 7 et entre 6 et 7, respectivement aux points $G$ et $H$ qui se font avec frottement. Le coefficient de frottement est de 0,15. On se placera à la limite du glissement qui correspond au cas extrême.

On pourra faire l'hypothèse que le problème est plan.

\paragraph{}
\textit{Déterminer si, pour serrer le frein, la haute pression dans le vérin doit se situer dans la cavité supérieure ou inférieure.}

\paragraph{}
\textit{La pression dans le vérin est de 200 bars. La section du vérin est de $30\; cm^3$. Déterminer l'effort que le vérin exerce sur 9 pour serrer le frein.}

\paragraph{}
\textit{Tracer le graphe d'analyse associé au système.}

\paragraph{}
\textit{Après avoir isolé le solide 2, appliquer le PFS et déterminer les inconnues de liaisons.}

\paragraph{}
\textit{Après avoir isolé les solides 3 et 4, appliquer le PFS et déterminer les inconnues de liaisons.}

\paragraph{}
\textit{Après avoir isolé le solide 9, appliquer le PFS et déterminer les inconnues de liaisons.}

\paragraph{}
\textit{Après avoir isolé le solide 5, appliquer le PFS et déterminer les inconnues de liaisons.}

\paragraph{}
\textit{Après avoir isolé le solide 6, appliquer le PFS et déterminer les inconnues de liaisons.}

\paragraph{}
\textit{On donne $||\vect{KG}||=||\vect{KH}||=12\;cm$. Déterminer le couple de freinage qui s'exerce sur les pièces 5, 6 et 7. }

\paragraph{}
\textit{On donne $||\vect{KJ}||=8\;cm$. Calculer le poids maximal de l'objet que le frein de sécurité peut freiner. Conclure quant à la capacité du frein de sécurité à satisfaire le niveau du critère de la fonction FS1.}


\begin{center}
\includegraphics[width=.55\textwidth]{png/grue2}
\end{center}


\begin{thebibliography}{2}
\bibitem{grue}{\url{http://www.fond-ecran-image.com/galerie-membre,geometrique,grue-portuaire-076jpg.php}}
\end{thebibliography}
\end{document}