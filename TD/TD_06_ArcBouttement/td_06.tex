\documentclass[11pt,oneside]{article}
\input{coursHeadings}

\usepackage[%
    pdftitle={TD Statique},
    pdfauthor={Xavier Pessoles},
    colorlinks=true,
    linkcolor=blue,
    citecolor=magenta]{hyperref}



% \makeatletter \let\ps@plain\ps@empty \makeatother
%% DEBUT DU DOCUMENT
%% =================
\sloppy
\hyphenpenalty 10000

\newcommand{\Pointilles}[1][3]{%
\multido{}{#1}{\makebox[\linewidth]{\dotfill}\\[\parskip]
}}


\begin{document}


\newboolean{prof}
\setboolean{prof}{false}
%------------- En tetes et Pieds de Pages ------------
\pagestyle{fancy}
\renewcommand{\headrulewidth}{0pt}

\fancyhead{}
\fancyhead[L]{%
\begin{minipage}[c]{1.6cm}
\includegraphics[width=1.4cm]{png/logo_jh_ptsi.png}%
\end{minipage}
\rule{2cm}{.5pt}
}

\fancyhead[C]{\rule{12cm}{.5pt}}

\fancyhead[R]{%
\begin{minipage}[c]{3cm}
\begin{flushright}
\footnotesize{\textit{\textsf{Sciences Industrielles\\ pour l'Ingénieur}}}%
\end{flushright}
\end{minipage}
}

\renewcommand{\footrulewidth}{0.2pt}

\fancyfoot[C]{\footnotesize{\bfseries \thepage}}
\fancyfoot[L]{\footnotesize{2011 -- 2012} \\ X. \textsc{Pessoles}}
\ifthenelse{\boolean{prof}}{%
\fancyfoot[R]{\footnotesize{TD -- CI 3 : Statique -- P}}
}{%
\fancyfoot[R]{\footnotesize{TD -- CI 3 : Statique}}
}


%\begin{center}
%\textit{Centre d'intérêt}
%\end{center}

\begin{center}
 \huge\textsc{CI 3 -- Statique : Modélisation, prévision et vérification du comportement statique des systèmes}
\end{center}

\begin{center}
 \LARGE\textsc{Chapitre 3 -- Étude graphique} 
\end{center}

%\setcounter{paragraph}{0}
\vspace{.5cm}


\section*{Remonte pente}

On s'intéresse à un remonte pente dont on donne la modélisation plane simplifiée ainsi qu'un extrait de cahier des charges fonctionnel.

\begin{center}
\includegraphics[width=.9\textwidth]{png/img1.png}
\end{center}

On suppose pour les besoins de l'étude que le câble 4 est à l'arrêt. La liaison entre la perche 2 et le fourreau 3 est une liaison pivot parfaite d'axe $(C,\vect{z})$. La liaison entre le fourreau 3 et le câble 4 est une liaison de type pivot glissant d'axe $(O,\vect{x})$ construite volontairement avec beaucoup de jeu de sorte que le fourreau puisse s'incliner sous l'effet de l'effort de traction du skieur et venir s'arc-bouter sur le câble 4. Dans ce cas, on considère que la liaison entre le câble 4 et le fourreau 3 correspond à une liaison ponctuelle avec frottement au point $A$ de normale $-\vect{y}$ et une liaison ponctuelle avec frottement au point $B$ de normale $\vect{y}$.

L'objectif est de vérifier le critère de performance de la fonction de service FS2. 

\paragraph{}
\textit{Montrer que le torseur d'action mécanique transmissible de 3 sur 4 au point A est un glisseur dont le support passe par A.}

\paragraph{}
\textit{Montrer que le torseur d'action mécanique transmissible par 3 sur 4 au point B est un glisseur dont le support passe par $B$. }

Par la suite, on notera $\vect{A_{(3\rightarrow 4)}}$ et $\vect{B_{(3\rightarrow 4)}}$ les résultantes de ces torseurs et $\Delta_A$ et $\Delta_B$ leurs supports.

\paragraph{}
\textit{Déterminer graphiquement sur le document réponse 1 pour un coefficient de frottement de 0,4 la zone $Z$ formée de l'ensemble des points où peut se trouver le point d'intersection des supports de ces deux glisseurs quand ces deux glisseurs remplissent les conditions d'équilibre statique de 3.}

\paragraph{}
\textit{Montrer que le torseur d'action mécanique transmissible de 2 sur 3 est un glisseur passant par $C$.}

\paragraph{}
\textit{On appelle $\Delta_{23}$ le support de ce glisseur et $\alpha$ l'angle compris entre ce support et l'axe du câble. Déterminer graphiquement cet angle et conclure vis-à-vis du cahier des charges pour la configuration correspondant à celle du document réponse 1.}


\begin{center}
\includegraphics[width=.7\textwidth]{png/img2.png}
\end{center}



\newpage


\setcounter{paragraph}{0}

\subsection*{Pince lève tôles}
On s'intéresse à une pince utilisée pour la saisie et le transport des plaques dont on donne la modélisation ainsi qu'un extrait de cahier des charges fonctionnel.

\begin{center}
\includegraphics[width=.9\textwidth]{png/img3.png}
\end{center}

Accroché à un portique, ce système est amené au dessus de la plaque à saisir $(2)$ puis 
puis abaissé de sorte que la plaque s'engage entre le flanc plat de 1 et la bille 3. Lorsque le préhenseur 1 est relevé, la bille 3 coince la plaque 2 contre le flanc. La plaque est alors contrainte à suivre le préhenseur dans ses déplacements.

Hypothèses : 
\begin{itemize}
\item la nature des mouvements (montée/descente à très faible vitesse) est telle que l'étude de la stabilité du système peut être abordée par une étude statique;
\item on utilise deux préhenseurs placés aux deux extrémités de la plaque. La symétrie du problème global permet de ramener l'étude à celle d'un préhenseur unique et de travailler dans le plan transversal de celui-ci (voir figure document réponse 1);
\item chaque préhenseur supporte la moitié de la masse de la plaque;
\item les poids de la bille 3 et du préhenseur 1 sont supposés négligeables devant les autres forces en présence.
\end{itemize}


\paragraph{}
\textit{Déterminer, en fonction de $\theta$, la valeur $f_{min}$ que doit avoir le coefficient de frottement en A et en B pour que l'ensemble puisse rester en équilibre en position bloquée.}

\paragraph{}
\textit{On suppose dans un premier temps qu'il n'y a pas de frottement entre 2 et 1. Déterminer graphiquement sur le document réponse 1 les actions mécaniques s'exerçant sur 2 pour la masse maximale correspondant au cahier des charges (Echelle : $1\;cm=1\,000\;N$, $g\simeq 10m\cdot s^{-2}$).}

\paragraph{}
\textit{On considère maintenant qu'il y a du frottement entre 2 et 1. Indiquer en raisonnant à partir de la construction graphique précédente si la présence de frottement est de nature à augmenter ou diminuer l'intensité des forces sur la bille 2.}

\begin{center}
\includegraphics[width=.7\textwidth]{png/img4.png}
\end{center}


\end{document}